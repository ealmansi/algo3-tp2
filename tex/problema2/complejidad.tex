Vamos a probar que nuestro algoritmo tiene una complejidad de $O(n^2)$ operaciones. Para eso vamos a analizar el pseudocódigo de a partes.

\begin{center}
\begin{pseudo}
\State Tipo de dato Pueblo es Tupla $\langle$ x : entero, y : entero $\rangle$
\State Tipo de dato Conexion es Tupla $\langle p_1 : Pueblo, p_2 : Pueblo \rangle$
    \Procedure{La centralita (de gas)}{$\langle Pueblo_1 , \ldots , Pueblo_n \rangle , k$}
        \State $Pueblos \leftarrow$ nuevo conjunto de Pueblo \Ode{1}
        \State Cargo\_Pueblos(Pueblos) \Ode{n}
        \State $Conexiones \leftarrow$ nuevo conjunto de Conexion \Ode{1}
        \State Conexiones = Prim(Pueblos) \Ode{n^2}
        \State Sort(Conexiones) \Ode{n*log(n)}
        \For {1 to k-1} \Ode{1}
	  \State Eliminar máximo Conexiones \Ode{1}
	\EndFor
        \State Centrales $\leftarrow$ nuevo lista de Pueblos \Ode{1}
        \State yaPase $\leftarrow$ vector $<$ bool $>$ de tamaño n + 1 \Ode{n}
        %TODO mejorar esta parte
        \For {cada p $\in$ Pueblos} \Ode{1}
        \If{!leLlegaGas(p)} \Ode{1}
        \State agregar p a Centrales \Ode{1}
        \Comment Esto es para poner que les llega gas a los pueblos.
        \State BFS(p,Pueblos, Conexiones, yaPase) \Ode{$n$}
        \EndIf
        \EndFor
        \State \textbf{return} Centrales,Conexiones \Ode{n}
    \EndProcedure
\end{pseudo}
\end{center}

Las incializaciones del princpio (son hasta Prim) tienen una complejidad temporal de $O(n)$ debido a que cargamos n pueblos, y cargar cada pueblo tiene una complejidad temporal constante. Además creamos dos conjuntos en una complejidad de $O(1)$.

Realizar Prim tiene un costo temporal de $O(n^2)$ ya que lo implementamos sobre listas.

Por otro lado, reordenar las conexiones cuesta $O(n*log(n))$ teniendo en cuenta que Prim nos va a dar n-1 conexiones, esto es debido a que Prim devuelve un árbol. Utilizamos el algoritmo \emph{Sort} de las librerías de C++ que tiene la complejidad mencionada \footnote{http://en.cppreference.com/w/cpp/algorithm/sort}.

Para eliminar las k-1 conexiones más grandes, simplemente las borramos de la lista de conexiones. Como están ordenadas, eliminar una conexión cuesta $O(1)$ por lo que eliminar k-1 conexiones cuesta $O(k)$. Como estamos contemplando casos donde k  $<$ n, también podemos decir que tiene un costo temporal de $O(n)$.

Luego, realizamos más inicializaciones que tienen un costo temporal total de $O(n)$, ya que creamos un vector de $n$ booleanos en $O(n)$ y creamos una lista de Pueblo en $O(1)$.

Ahora vamos a pasar a analizar el ciclo \emph{for} del final. Para cada pueblo preguntamos si le llega gas o no. En el caso en que sí le llegue gas, pasamos al siguiente. En el caso en que no, colocamos una central en dicho pueblo, y recorremos sus vecinos para recordar que a todos ellos les llega gas.

La guarda del if es true k veces (una por cada componente conexa) ya que luego de sacar $k$ conexiones del árbol que nos devuelve Prim obtenemos $k$ componentes conexas (demostraremos eso en la próxima sección. Como sólo realizamos el BFS si entra a la guarda, podemos afirmar que la complejidad temporal del ciclo \emph{for} es de $O(n*k)$. Recordando que k $<$ n, podemos decir que tiene una complejidad temporal de $O(n^2)$.

Finalmente, devolvemos las centrales y las conexiones que cuesta $O(1)$.

Por álgebra de órdenes, la complejidad total de nuestro algoritmo es: $O(1)$ + $O(n)$ + $O(n^2)$ + $O(n*log(n))$ = $O(n^2)$. Omitimos las constantes que multiplican a los órdenes, ya que no cambia la complejidad temporal asintótica y las consideramos despreciables.