Primero vamos a demostrar que un bosque generador mínimo con k componentes conexas resuelve el problema. Osea que todos los pueblos estan conectados y su máxima arista es la mínima entre todas las soluciones posibles.

Para eso primero vamos a demostrar los siguiente lemas:

\textbf{Lema 1:} Un árbol generador mínimo tambien minimiza la máxima arista.

\textbf{Lema 2:} G grafo es solucion $\Leftrightarrow \#$ (componentes conexas (G)) $\leq$ k. Osea,  G(V,E) grafo y C un conjunto de k vertices $\in $ V q.v.q. $(\forall v \in V)(\exists c \in C)$ v esta conecado a c $\Leftrightarrow \#$ (componentes conexas (G)) $\leq$ k

\textbf{Lema 3:} Dada una solucions con C componentes conexas (C $<$ k), existe una solucion mejor o igual con C+1 componentes conexas.


\textbf{Lema 1:} Un árbol generador mínimo tambien minimiza la máxima arista.

Sea G(V,E) un árbol generador mínimo y e su máxima arista.

Sea G'(V,E-e), tiene dos componentes conexas C1 y C2 por ser G un árbol.

Por el absurdo supongo que existe otro grafo conexo H(V,E') cuya arista máxima es e' y f(e') $<$ f(e).
Como H es conexa tiene que existir algun e'' tal que un extremo pertenezca a C1 y el otro a C2. y f(e'') $\leq$ f(e') ya que e'es la máxima.

Sea G''(V,E-e+e'') G'' es árbol ya que e'' tiene un extremo en cada componente conexa q se forma al quitar e, osea q sigue siendo conexo y la cantidad de arista se mantiene en m = n-1. Sea l la arista de mayor peso de G''.

Sabemos que $f(e'') < f(e) \Rightarrow f(e'') - f(e) < 0$

$\sum_{i \in E-e+e''} f(i) = \sum_{i \in E} f(i) - f(e) + f(e'') < \sum_{i \in E} f(i)$.

Tambien sabemos que $\sum_{i \in E} f(i) \leq \sum_{i \in E-e+e''} f(i)$ ya que G es un árbol generador mínimo.

Entonces: $\sum_{i \in E} f(i) > \sum_{i \in E-e+e''} f(i) \wedge \sum_{i \in E} f(i) \leq \sum_{i \in E-e+e''} f(i)$
Lo cual es absurdo.

\textbf{Lema 2:} G(V,E) grafo y C un conjunto de k vertices $\in $ V q.v.q. $(\forall v \in V)(\exists c \in C)$ v esta conecado a c $\Leftrightarrow \#$ (componentes conexas (G)) $\leq$ k

Sea k' la cantidad de $\#$ (componentes conexas (G)).

Primero probemos $\Leftarrow$

Tomo $c_i \in S_i$ con $c_i$ el i-esimo vertice de C y $S_i$ la i-esima componente conexa de S para $1 \leq i \leq k'$ puedo elegir k' elemento porque C tiene k elementos y k' $\leq$ k.

Ahora tomo cualquier v $\in$ V, sea $S_j$ la componente conexa de G a la que v pertenece. Existe un camino entre v y $c_j$ ya que ambos pertenecen a la misma componente conexa.

Ahora probemos $\Rightarrow$
Para esto probaremos el contra reciproco:

$\#$ (componentes conexas (G)) $>$ k  $\Leftrightarrow$ $(\exists v \in V)(! \exists c \in C)$ v esta conecado a c. 

Sea ${v_1,...,v_{k'}}$ un conjuntos de vertices de V tal que $v_i \in S_i$. $(\forall 1 \leq i,j \leq k', i \neq j)$ no existe camino entre $v_i$ y $v_j$, ya que pertenecen a componentes conexas diferentes.

Ahora tomo cada uno de los $c_i \in$ C y los asocio al $v_j$ / exista un camino entre $c_i$ y $v_j$. Veamos que un $c_i$ no puede estar asociado a dos v diferentes. ya que si existe un camino entre $c_i$ y $v_j$ y tambien entre $c_i$ y $v_q$ tambien existira el camino entre $v_j$ y $v_q$ lo cual es absurdo salvo que j = q.

Cuando terminamos de asignar los k $c_i$ tendremos a lo sumo (ya que dos c diferentes si pueden corresponder al mismo v) k $v_i$ con un c asignado. Pero como k $<$ k' tendremos como mínimo un $v_i$ que no esta conectado a ningun c. Demostrando asi el lema.

\textbf{Lema 3:} Dada una solucions con C componentes conexas (C $<$ k), existe una solucion mejor o igual con C+1 componentes conexas.

Tomo una solución G(V,E) con C componentes conexas (C $<$ k), Sea e $\in$ E y G'(V,E-e).
Primero veamos que G' es solucion, como G tenia C componentes conexas, G' tiene a lo sumo C+1. Como C $<$ k $\Rightarrow$ C+1 $\leq$ k. Por el \textbf{Lema 2} G' es solución. 
Ahora quiero ver que es mejor o igual que G.
G' es mejor o igual que G $\Leftrightarrow$ MaxArista(G') $\leq$ MaxArista(G).
Como las aristas de G' estan incluidas en las de G, no puede existir una arista de G' que sea mayor a la máxima arista de G.
Entonces probamos G' es mejor o igual que G.