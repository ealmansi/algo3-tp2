Primero vamos a demostrar que un bosque generador mínimo con k componentes conexas resuelve el problema. Osea que todos los pueblos estan conectados y su máxima arista es la mínima entre todas las soluciones posibles.

Para eso primero vamos a demostrar los siguiente lemas:

\textbf{Lema 1:} Un bosque de k componentes conexas generador mínimo tambien minimiza la máxima arista.

\textbf{Lema 2:} G grafo es solucion $\Leftrightarrow \#$ (componentes conexas (G)) $\leq$ k. Osea,  G(V,E) grafo y C un conjunto de k vertices $\in $ V q.v.q. $(\forall v \in V)(\exists c \in C)$ v esta conecado a c $\Leftrightarrow \#$ (componentes conexas (G)) $\leq$ k

\textbf{Lema 3:} Dada una solucion con C componentes conexas (C $<$ k), existe una solucion mejor o igual con C+1 componentes conexas. \\ \\ 


\textbf{Lema 1:} Un bosque de k componentes conexas generador mínimo tambien minimiza la máxima arista. Equivalentemente, sea B un bosque de k componentes conexas generador mínimo: $(\nexists$ G subgrafo generador con k componentes conexas) $MaxArista(G) < MaxArista(B)$.

Vamos a demostrar este lema por el absurdo, sea B(V,E) un bosque de k componentes conexas generador mínimo supongo que existe un G(N,A) subgrafo generador con k componentes conexas tal que $MaxArista(G) < MaxArista(B)$.

Sea e la MaxArista(B), sabemos que toda arista en en G va a ser menor que e.

Si las k componentes conexas de B y G son las mismas:

Sea B'(V,E-e), tiene k-1 componentes conexas iguales a las de B, y C,la componente en la que estaba e, se divide en dos componentes conexas C1 y C2, ambas árboles.

Como G tiene las mismas componentes conexas que B, C es una componente conexa de G.Por lo cual existe una arista e' con un extremo en C1 y otro en C2.

Sea B''(V,E-e+e') B'' es un subgrafo generador con k componentes conexas ya que al quitar e se generan un subgrafo con k+1 componentes conexas. Y al agregar e', como este  tiene extremos en dos componente conexa diferentes, quedan k componentes conexas. Y como no se modifican los vertices sigue siendo generador.

Sabemos que $f(e') < f(e) $por e' pertenecer a G $\Rightarrow f(e') - f(e) < 0$

$\sum_{i \in E-e+e'} f(i) = \sum_{i \in E} f(i) - f(e) + f(e') < \sum_{i \in E} f(i)$.

Tambien sabemos que $\sum_{i \in E} f(i) \leq \sum_{i \in E-e+e'} f(i)$ ya que B es un bosque de k componentes conexas generador mínimo.

Entonces: $\sum_{i \in E} f(i) > \sum_{i \in E-e+e'} f(i) \wedge \sum_{i \in E} f(i) \leq \sum_{i \in E-e+e'} f(i)$
Lo cual es absurdo.

El otro caso es que G y B tengan aunque sea dos componentes conexas diferentes:

Si las componentes conexas de G son diferentes a las de B, entonces existe en G una arista e' tal que cada extremo de la arista pertenece a componentes conexas diferentes de B (Sino terndría las mismas componentes conexas).

Sea B''(V,E-e+e') igual que en el caso anterior B'' es un subgrafo generador con k componentes conexas.

Y de la misma manera que en el caso anterior se llega a un absurdo.

abemos que $f(e') < f(e) $por e' pertenecer a G $\Rightarrow f(e') - f(e) < 0$

$\sum_{i \in E-e+e'} f(i) = \sum_{i \in E} f(i) - f(e) + f(e') < \sum_{i \in E} f(i)$.

Tambien sabemos que $\sum_{i \in E} f(i) \leq \sum_{i \in E-e+e'} f(i)$ ya que B es un bosque de k componentes conexas generador mínimo.

Entonces: $\sum_{i \in E} f(i) > \sum_{i \in E-e+e'} f(i) \wedge \sum_{i \in E} f(i) \leq \sum_{i \in E-e+e'} f(i)$. \\ \\


\textbf{Lema 2:} G(V,E) grafo y C un conjunto de k vertices $\in $ V q.v.q. $(\forall v \in V)(\exists c \in C)$ v esta conecado a c $\Leftrightarrow \#$ (componentes conexas (G)) $\leq$ k

Sea k' la cantidad de componentes conexas de G.

Primero probemos $\Leftarrow$

Tomo $c_i \in S_i$ con $c_i$ el i-esimo vertice de C y $S_i$ la i-esima componente conexa de G para $1 \leq i \leq k'$ puedo elegir k' elementos porque C tiene k elementos y k' $\leq$ k.

Ahora tomo cualquier v $\in$ V, sea $S_j$ la componente conexa de G a la que v pertenece. Existe un camino entre v y $c_j$ ya que ambos pertenecen a la misma componente conexa.

Ahora probemos $\Rightarrow$
Para esto probaremos el contra reciproco:

$\#$ (componentes conexas (G)) $>$ k  $\Leftrightarrow$ $(\exists v \in V)(! \exists c \in C)$ v esta conecado a c. 

Sea ${v_1,...,v_{k'}}$ un conjuntos de vertices de V tal que $v_i \in S_i$. $(\forall 1 \leq i,j \leq k', i \neq j)$ no existe camino entre $v_i$ y $v_j$, ya que pertenecen a componentes conexas diferentes.

Ahora tomo cada uno de los $c_i \in$ C y los asocio al $v_j$ / exista un camino entre $c_i$ y $v_j$. Veamos que un $c_i$ no puede estar asociado a dos v diferentes. ya que si existe un camino entre $c_i$ y $v_j$ y tambien entre $c_i$ y $v_q$ tambien existira el camino entre $v_j$ y $v_q$ lo cual es absurdo salvo que j = q.

Cuando terminamos de asignar los k $c_i$ tendremos a lo sumo (ya que dos c diferentes si pueden corresponder al mismo v) k $v_i$ con un c asignado. Pero como k $<$ k' tendremos como mínimo un $v_i$ que no esta conectado a ningun c. Demostrando asi el lema. \\ \\


\textbf{Lema 3:} Dada una solucions con C componentes conexas (C $<$ k), existe una solucion mejor o igual con C+1 componentes conexas.

Tomo una solución G(V,E) con C componentes conexas (C $<$ k), Sea e $\in$ E y G'(V,E-e).

Primero veamos que G' es solucion. Como G tenia C componentes conexas, G' tiene a lo sumo C+1. Como C $<$ k $\Rightarrow$ C+1 $\leq$ k. Por el \textbf{Lema 2} G' es solución.

Ahora quiero ver que es mejor o igual que G.

Recordemos que, G' es mejor o igual que G $\Leftrightarrow$ MaxArista(G') $\leq$ MaxArista(G).

Como las aristas de G' estan incluidas en las de G, no puede existir una arista de G' que sea mayor a la máxima arista de G.

Entonces probamos G' es mejor o igual que G. \\ \\

Ahora utilizando los lemas 2 y 3, se llega a que siempre existe una solución optima con k componentes conexas.
Ya que por 2 si tiene menos que k no puede ser solución, y por 3 si tiene mas que k siempre se puede encontrar una mejor o igual con k.

Ahora usando el lema uno sabemos que entre todas las soluciones posibles con k componentes conexas, el bosque generador mínimo es óptimo.

Por lo cual concluimos que el bosque generador mínimo con k componentes conexas es una solución óptima para nuestro problema.