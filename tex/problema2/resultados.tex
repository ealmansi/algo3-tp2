Dado que el cómputo de la pérdida total tiene un costo fijo determinado por la cantidad de piezas, la variación en el tiempo de ejecución depende exclusivamente del costo de ordenamiento. Por lo tanto, no nos es posible determinar \emph{a priori} si un caso va a resultar mejor o peor para nuestra implementación, sin realizar un complejo análisis de peor caso sobre la función de ordenamiento.

A su vez, la predominancia de la etapa de ordenamiento implica que no va a ser posible mejorar la cota de complejidad teórica de nuestra solución, debido a que los algoritmos de ordenamiento por comparaciones son $\Omega(n * log(n))$.