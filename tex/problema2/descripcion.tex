Dada una serie de pueblos con sus respectivas ubicaciones, se desea determinar un plan de construcción que permita abastecer a todos ellos con gas natural. Es posible construir una cantidad limitada de centrales distribuidoras de gas, cada una de ellas en un pueblo, así como tuberías uniendo las distintas ubicaciones. Un pueblo se considera abastecido de gas si existe un camino por medio de tuberías desde el mismo hasta alguna central distribuidora, incluso si dicho camino pasa por otros pueblos. Por ejemplo, si el pueblo $p_1$ tiene una central y los pueblos $p_2$ y $p_3$ no, se puede construir una tubería del pueblo $p_1$ al pueblo $p_2$ y una del pueblo $p_2$ al pueblo $p_3$, quedando abastecidos de esta forma los tres pueblos sin necesidad de realizar una tubería directa entre $p_1$ y $p_3$.

Adicionalmente, se define el riesgo asociado a un plan de construcción como la mayor longitud de todas las tuberías que se deben construir para realizar el mismo. Entre todos los posibles planes que abastecen a todos los pueblos, se desea hallar aquel que tenga un riesgo asociado mínimo (o cualquiera de ellos, si hay más de uno).

La ubicación de cada pueblo queda determinada por sus coordenadas $x$ e $y$, por lo cual si tenemos $n$ pueblos y podemos construir a lo sumo $k$ centrales, cada instancia del problema se puede codificar de la siguiente manera:

\begin{verbatim}
  n k
  x1 y1
  x2 y2
  ...
  xn yn
\end{verbatim}

Luego, lo que buscamos es distribuir a lo sumo $k$ centrales entre los $n$ pueblos y realizar las conexiones entre ellos que sean necesarias para que todos queden abastecidos, de forma tal que se minimice la máxima longitud entre todas las tuberías construidas. Para representar una solución, alcanza con determinar la cantidad $q$ de centrales construídas, la cantidad $m$ de tuberías construídas, el índice de los pueblos donde se construyeron las centrales $c_1, c_2, ... c_q$, y los índices $i$, $j$ de los pueblos conectados por cada tubería:

\begin{verbatim}
  q m
  c1 c2 ... cq
  i1 j1
  ...
  im jm
\end{verbatim}

\subsubsection{Ejemplos y observaciones}

A continuación incluímos un ejemplo donde se tienen $10$ pueblos y se pueden construir a lo sumo $3$ centrales.
\begin{verbatim}
  10 3
  1 1
  1 2
  2 1
  7 8
  9 10
  8 7
  25 26
  27 28
  28 28
  27 27
\end{verbatim}

Dado que los pueblos se encuentran agrupados en tres grupos marcados $(1, 2, 3)$, $(4, 5, 6)$ y $(7, 8, 9, 10)$, es razonable que una solución óptima como la presentada aqui debajo consista en colocar una central por grupo, evitando así tuberías largas que unan pueblos alejados:

\begin{verbatim}
  3 7
  1 4 7 
  5 4
  10 7
  6 4
  2 1
  3 1
  8 10
  9 8
\end{verbatim}

Destacamos como observación el hecho de que no es una prioridad la minimización de la cantidad de centrales ni la cantidad de tuberías construidas, y por lo tanto dada una solución óptima, cualquier otra con mayor cantidad de construcciones será equivalente mientras tenga un mismo riesgo asociado.