Este problema trata de varios pueblos a los que hay que hacerles llegar gas natural. Para esto se colocarán centrales en algunos pueblos, y se construirán tuberías desde estos pueblos a los otros pueblos a los que no se les ponga una central. 

Estas tuberías no deben ir necesariamente de luna central al pueblo de forma directa, por ejemplo si el pueblo 1 tiene una central y los pueblos 2 y 3 no las tienen, se puede hacer una tubería del pueblo 1 al pueblo 2 y una del pueblo 2 al pueblo 3, de esta forma no se necesita una tubería que vaya directamente del pueblo 1 al pueblo 3.

La ubicación de los esta determinada por una coordenada $x$ y una coordenada $y$.

Tenemos $k$ centrales que debemos distribuir en $n$ pueblos. El problema es que mientras más largas sean las tuberías hay un mayor riesgo de que se produzca una fuga de gas.

Lo que se nos pide es distribuir las $k$ centrales en los $n$ pueblos minimizando la máxima longitud de tuberías entre un pueblo y otro.

\subsubsection{Ejemplos y observaciones}

