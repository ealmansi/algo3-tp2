Modelamos este problema de la siguiente manera:

$\bullet$ Un pueblo es un par ordenado (x,y).

$\bullet$ Una tubería es un par no ordenado de pueblos.

$\bullet$ Un conjunto de V pueblos y otro E de tuberías puede ser considerado un grafo, cuyo conjunto de nodos es V y su conjunto de aristas es E.

$\bullet$ El peso asociado a una arista (tubería) es igual a la distancia euclidiana entre los extremos (pueblos).

$\bullet$ Una central es un pueblo.

$\bullet$ Dado V un conjunto de pueblos y k la cantidad de centrales de gas, una solución es un C subconjunto de k pueblos de V y un grafo B(V,E), sea E el conjunto de tuberìas que decidimos colocar para conectar los pueblos. Si queremos conectar al pueblo 'a' y al pueblo 'b', agregamos al conjunto E la arista ('a','b'). El subonjunto C representa a los pueblos donde elegimos poner las centrales.  

$\bullet$ Dada una solución conformada por C y B(V,E), un pueblo $v \in V$ esta provisto de gas si existe un camino en B entre éste y un pueblo $c \in C$ con central distribuidora.

$\bullet$ Una solución C, B(V,E) es factible si $(\forall v \in V)(\exists c \in C) \exists$ un camino entre v y c en B. 

$\bullet$ Queremos encontrar el C , B(V,E) tal que sea una solución óptima, minimizando la máxima arista.

Consideramos una solución, si cumple la aridad pedida. Una solución factible si cumple cierta condicion además de la aridad. Una solución optima es quien maximiza/minimiza cierta funcion entre todas las soluciones factibles, en este caso se busca minimizar la máxima arista.

Para encontrar esto vamos a buscar B(V,E) un bosque generador mínimo con k componentes conexas de G el grafo completo generado por los vértices de V y C tenga un pueblo de cada una de las componentes conexas de G.

Definiendo un bosque generador mínimo con k componentes conexas de G, como un grafo que cumple las siguientes condiciones:

$\bullet$Es un subgrafo generador de G. 

$\bullet$Es un bosque de k componentes conexas.

$\bullet$ Entre todos los subgrafos generador de G de k componentes conexas tiene la mínima suma de aristas.

Ahora presentaremos una idea de por qué funciona. Luego vamos a demostrar formalmente que esto resuelve el problema. 

$\bullet$ Un bosque de k componentes conexas generador mínimo también minimiza la máxima arista.

$\bullet$ Dada una solución C, B(V,E), es factible $\Longleftrightarrow$ $(\forall D componente conexa de B)(\exists v \in D) v \in C$.  

$\bullet$ Dada una solución con i componentes conexas (siendo i $<$ k), existe una solución mejor o igual con i+1 componentes conexas.

Vamos a suponer que k $\leq$ n, ya que en el caso contrario la solución es trivial. Cada pueblo tiene una central y no se necesitan construir ninguna tubería.