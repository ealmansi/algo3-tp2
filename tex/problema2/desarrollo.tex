Modelamos este problema de la sigueinte manera:

$\bullet$ Un pueblo es un par ordenado (x,y).

$\bullet$ Que dos pueblos sean adyacentes, significa que están conectados.

$\bullet$ El peso de la arista entre los pueblos $p_1$ y $p_2$ es igual a la distancia euclidiana entre ambos.

$\bullet$ Un pueblo esta provisto de gas si existe un camino entre éste y un pueblo con central distribuidora.

$\bullet$ Tenemos k centrales de gas, las cuales van a ser colocadas en algún pueblo. Queremos minimizar los pesos de las aristas de forma tal que el peso máximo, sea lo mínimo posible %TODO no se como escribir lo del minimo maximo. (Santi) Asi va?

Para encontrar esto vamos a buscar un bosque generador mínimo con k componentes conexas. Ahora presentaremos una idea de por qué funciona. Luego vamos a demostrar formalmente que esto resuelve el problema.

$\bullet$ Un árbol generador mínimo también minimiza el peso máximo entre las aristas.

$\bullet$ Teniendo k centrales, puedo tener a lo sumo k componentes conexas, sino sera imposible proveer de gas a todos los pueblos.

$\bullet$ Dada una solución con C componentes conexas (siendo C $<$ k), existe una solución mejor o igual con C+1 componentes conexas.

Vamos a suponer que k $\leq$ n, ya que en el caso contrario la solucion es trivial. Cada pueblo tiene una central y no se necesitan construir ninguna tubería.