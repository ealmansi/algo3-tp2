Modelamos este problema de la sigueinte manera:

$\bullet$ Un pueblo es un par ordenado (x,y).

$\bullet$ Que dos pueblos esten conectados es que sean adyacentes.

$\bullet$ El peso de la arista entre los pueblos p1 y p2 es igual a la distancia euclidiana entre ambos.

$\bullet$ Un pueblo esta provisto de gas si existe un camino entre el y un pueblo con central distribuidora.

$\bullet$ Queremos hayar conjunto de pares no ordenados de pueblos y un conjunto de k pueblos / %TODO no se como escribir lo del minimo maximo


Para encontrar esto vamos a buscar un bosque generador mínimo con k componentes conexas.
Aunque lo demostraremos después esto esta basado en lo siguiente:

$\bullet$ Un árbol generador mínimo tambien minimiza la máxima arista.

$\bullet$ Teniendo k centrales, puedo tener a lo sumo k componentes conexas, sino sera imposible proveer de gas a todos los pueblos

$\bullet$ Dada una solucions con C componentes conexas (C < k), existe una solucion mejor o igual con C+1 componentes conexas.

