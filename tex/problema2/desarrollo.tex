Modelamos este problema de la siguiente manera:

$\bullet$ Un pueblo es un par ordenado (x,y).

$\bullet$ Que dos pueblos sean adyacentes, significa que están conectados.

$\bullet$ El peso de la arista entre los pueblos $p_1$ y $p_2$ es igual a la distancia euclidiana entre ambos.

$\bullet$ Un pueblo esta provisto de gas si existe un camino entre éste y un pueblo con central distribuidora.

$\bullet$ Dado V un conjunto de pueblos y k la cantidad de centrales de gas, una solución es un C subconjunto de k pueblos de V y un grafo B(V,E).

$\bullet$Una solución C, B(V,E) es valida si $(\forall v \in V)(\exists c \in C) \exists$ un camino entre v y c. 

$\bullet$ Queremos encontrar el C , B(V,E) tal que sea una solución valida y entre todas las soluciones validas tenga la mínima máxima arista.

Para encontrar esto vamos a buscar B(V,E) un bosque generador mínimo con k componentes conexas de G el grafo completo generado por los vértices de V y C tenga un pueblo de cada una de las componentes conexas de G.

Definiendo un bosque generador mínimo con k componentes conexas de G, como un grafo que cumple las siguientes condiciones:

$\bullet$Es un subgrafo generador de G. 

$\bullet$Es un bosque de k componentes conexas.

$\bullet$ Entre todos los subgrafos generador de G de k componentes conexas tiene la mínima suma de aristas.

Ahora presentaremos una idea de por qué funciona. Luego vamos a demostrar formalmente que esto resuelve el problema. 

$\bullet$ Un bosque de k componentes conexas generador mínimo también minimiza la máxima arista.

$\bullet$ Dada una solución C, B(V,E), es valida $\Longleftrightarrow$ $(\forall D componente conexa de B)(\exists v \in D) v \in C$.  

$\bullet$ Dada una solución con i componentes conexas (siendo i $<$ k), existe una solución mejor o igual con i+1 componentes conexas.

Vamos a suponer que k $\leq$ n, ya que en el caso contrario la solución es trivial. Cada pueblo tiene una central y no se necesitan construir ninguna tubería.