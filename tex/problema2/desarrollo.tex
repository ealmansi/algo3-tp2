\subsubsection{Modelo formal}

Dado que los pueblos se pueden identificar según su índice, y que las tuberías son vínculos no dirigidos entre dos pueblos, resulta natural representar un plan de construcción mediante un grafo. Para ello, dado un plan particular se lo puede relacionar con el grafo $B(V, E)$, donde $V = \{1,\;2,\;...,\;n\}$ y vale que $(i,j) \in E$ sii se construye una tubería entre $p_i$ y $p_j$. Es necesario definir también un conjunto $C \subseteq V$, $|C| \leq k$, incluyendo los índices de aquellos pueblos donde se planea construir las centrales.

A las aristas de dicho grafo se les debe asignar una función de peso $w:E\rightarrow\mathbb{R}$, determinando la longitud de las tuberías que representan; concretamente:
$$si\;e = (i,j),\;w(e) = dist(p_i,p_j) = \sqrt{(x_i - x_j)^2 + (y_i - y_j)^2}$$

donde $dist$ representa la distancia euclidiana entre $p_i$ y $p_j$.

Modelando de esta forma un plan de construcción, un determinado plan $P = <C,\;B(V, E),\;w>$ debe cumplir las siguientes características para ser solución al problema:

\begin{itemize}
  \item Todo pueblo debe estar abastecido de gas. Es decir, $(\forall v \in V)(\exists c \in C)\;\exists$ un camino entre $v$ y $c$ en $B$. En este caso, decimos que el plan es una solución \emph{factible}.
  
  \item Dado cualquier otro plan factible $P'$, su riesgo asociado es mayor o igual al de $P$. Equivalentemente, si $P' = <C',\;B'(V', E'),\;w'>$ y $P'$ es factible, vale que $max\{w(e)\;|\;e\in E\} \leq max\{w'(e')\;|\;e'\in E'\}$. En este caso, decimos que el plan es una solución \emph{óptima}.
\end{itemize}

\subsubsection{Descripción del algoritmo}

En primer lugar, definimos un \emph{bosque generador mínimo con $k$ componentes conexas} de un grafo $G$ como un grafo que cumple las siguientes condiciones:

\begin{itemize}
  \item Es un subgrafo generador de $G$.

  \item Es un bosque de $k$ componentes conexas.

  \item Entre todos los subgrafos generadores de $G$ de $k$ componentes conexas tiene la mínima suma de aristas.
\end{itemize}

El algoritmo desarrollado para hallar la solución consiste en obtener el bosque generador mínimo con $k$ componentes conexas del grafo $G(V,E)$, donde $G$ representa el plan donde se contruyen todas las posibles tuberías entre todos los pueblos, siendo G de esta forma un grafo completo. Denominando $B(V,E')$ al bosque obtenido, el conjunto $C$ de la solución se puede conformar tomando un pueblo de cada una de las componentes conexas de $B$.

Ahora presentaremos una idea de por este modelo efectivamente resuelve el problema. Luego lo vamos a demostrar formalmente. 

\begin{itemize}
\item Un bosque de $k$ componentes conexas generador mínimo también minimiza la máxima arista.

\item Dada una solución $<C,\;B(V,E')>$, la misma es factible sii $C$ contiene un y solo un vértice de cada componente conexa de $B$. Formalmente, sii $(\forall D$ componente conexa de $B)((\exists v \in D) v \in C)$.  

\item Dada una solución con i componentes conexas (siendo i $<$ k), existe una solución mejor o igual (osea con un riesgo asociado menor o igual) con i+1 componentes conexas (no necesariamente factibles).
\end{itemize}

Vamos a suponer que k $\leq$ n, ya que en caso contrario se puede construir una central en cada pueblo sin necesidad de tuberías, y la solución es trivial.