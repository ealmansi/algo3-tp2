\documentclass[a4paper, 10pt, twoside]{article}

\usepackage[top=1in, bottom=1in, left=1in, right=1in]{geometry}
\usepackage[utf8]{inputenc}
\usepackage[spanish, es-ucroman, es-noquoting]{babel}
\usepackage{setspace}
\usepackage{fancyhdr}
\usepackage{lastpage}
\usepackage{amsmath}
\usepackage{amsfonts}
\usepackage{amsthm}
\usepackage{verbatim}
\usepackage{graphicx}
\usepackage{float}
\usepackage[noend]{algpseudocode}
\usepackage{enumitem} % Provee macro \setlist
\usepackage[toc, page]{appendix}
\usepackage{amsthm}
\usepackage{epstopdf}
\usepackage{amssymb}
\usepackage{caption}
\usepackage{subcaption}

%%%%%%%%%% Configuración de amsthm %%%%%%%%%%

\newtheorem{propiedad}{Propiedad}
\newtheorem{demostracion}{Demostración de la Propiedad }

%%%%%%%%%% Configuración de Fancyhdr - Inicio %%%%%%%%%%
\pagestyle{fancy}
\thispagestyle{fancy}
\lhead{Trabajo Práctico 2 · Algoritmos y Estructuras de Datos III}
\rhead{Aboy · Almansi · Canay · Decroix}
\renewcommand{\footrulewidth}{0.4pt}
\cfoot{\thepage /\pageref{LastPage}}

\fancypagestyle{caratula} {
   \fancyhf{}
   \cfoot{\thepage /\pageref{LastPage}}
   \renewcommand{\headrulewidth}{0pt}
   \renewcommand{\footrulewidth}{0pt}
}
%%%%%%%%%% Configuración de Fancyhdr - Fin %%%%%%%%%%


%%%%%%%%%% Configuración de Algorithmic - Inicio %%%%%%%%%%
% Entorno propio para customizar la presentación del pseudocódigo
\newenvironment{pseudo}[1][]{%
    \vspace{0.5em}%
    \begin{algorithmic}%
}
{%
    \end{algorithmic}%
    \vspace{0.5em}%
}

% Valores de verdad
\newcommand{\True}{\textbf{true}}
\newcommand{\False}{\textbf{false}}

% Conectivo 'in' para usar así: \ForAll{$foo$ \In $bar$}
\newcommand{\In}{\textbf{in} }

% Conectivo 'to' para usar así: \For{$i = 1$ \In $n$}
\newcommand{\To}{\textbf{to} }

% Complejidades
\newcommand{\Ode}[1]{\hfill $O(#1)$}
%%%%%%%%%% Configuración de Algorithmic - Fin %%%%%%%%%%


%%%%%%%%%% Miscelánea - Inicio %%%%%%%%%%
% Evita que el documento se estire verticalmente para ocupar el espacio vacío
% en cada página.
\raggedbottom

% Deshabilita sangría en la primer línea de un párrafo.
\setlength{\parindent}{0em}

% Separación entre párrafos.
\setlength{\parskip}{0.5em}

% Separación entre elementos de listas.
\setlist{itemsep=0.5em}

% Asigna la traducción de la palabra 'Appendices'.
\renewcommand{\appendixtocname}{Apéndices}
\renewcommand{\appendixpagename}{Apéndices}
%%%%%%%%%% Miscelánea - Fin %%%%%%%%%%


%%%%%%%%%% Gráficos - Inicio %%%%%%%%%%
% Macro para incluir tres gráficos (dentro de una figura) de manera que
% entren todos en una sola página.
\newcommand{\tresgraficos}[3]{
    \newcommand{\separacion}{-2.2em}
    \vspace{\separacion}
    \include{#1}
    \vspace{\separacion}
    \include{#2}
    \vspace{\separacion}
    \include{#3}
}
%%%%%%%%%% Gráficos - Fin %%%%%%%%%%


\begin{document}


%%%%%%%%%%%%%%%%%%%%%%%%%%%%%%%%%%%%%%%%%%%%%%%%%%%%%%%%%%%%%%%%%%%%%%%%%%%%%%%
%% Carátula                                                                  %%
%%%%%%%%%%%%%%%%%%%%%%%%%%%%%%%%%%%%%%%%%%%%%%%%%%%%%%%%%%%%%%%%%%%%%%%%%%%%%%%

\thispagestyle{caratula}

\begin{center}

\includegraphics[width=0.6\textwidth]{./img/DC.jpg} 
% \includegraphics[width=0.3\textwidth]{./img/UBA.jpg} 
\hfill

\vspace{2cm}

\begin{Huge}
Trabajo Práctico 2
\end{Huge}

\vspace{0.5cm}

\begin{Large}
Algoritmos y Estructuras de Datos III
\end{Large}

\vspace{1cm}

\begin{Large}
Primer Cuatrimestre de 2014
\end{Large}

\vspace{2cm}

\begin{tabular}{|c|c|c|}
\hline
Alumno & LU & E-mail\\
\hline
Aboy Solanes, Santiago    & 175/12 & santiaboy2@hotmail.com\\
Almansi, Emilio Guido     & 674/12 & ealmansi@gmail.com\\
Canay, Federico José      & 250/12 & fcanay@hotmail.com\\
Decroix, Facundo Nicolás  & 842/11 & fndecroix92@hotmail.com\\
\hline
\end{tabular}

\vspace{4cm}

Departamento de Computación,\\
Facultad de Ciencias Exactas y Naturales,\\
Universidad de Buenos Aires

\end{center}

\newpage


%%%%%%%%%%%%%%%%%%%%%%%%%%%%%%%%%%%%%%%%%%%%%%%%%%%%%%%%%%%%%%%%%%%%%%%%%%%%%%%
%% Índice                                                                    %%
%%%%%%%%%%%%%%%%%%%%%%%%%%%%%%%%%%%%%%%%%%%%%%%%%%%%%%%%%%%%%%%%%%%%%%%%%%%%%%%

\tableofcontents

\newpage


%%%%%%%%%%%%%%%%%%%%%%%%%%%%%%%%%%%%%%%%%%%%%%%%%%%%%%%%%%%%%%%%%%%%%%%%%%%%%%%
%% Introducción                                                              %%
%%%%%%%%%%%%%%%%%%%%%%%%%%%%%%%%%%%%%%%%%%%%%%%%%%%%%%%%%%%%%%%%%%%%%%%%%%%%%%%

\section{Introducción}
El objetivo de este informe es describir, desarrollar y presentar una solución algorítmica a tres problemas de maximización/minimización u optimización. Por otro lado, demostraremos la correctitud de las soluciones propuestas, y que su complejidad temporal cumple los requerimientos pedidos. Realizamos diversos experimentos que permiten verificar la correctitud, así como también realizamos experimentaciones computacionales para medir la performance de la implementación de nuestra solución. Los resultados obtenidos y la discusión de los mismos se encuentran en sus secciones correspondientes.

El código fuente de las soluciones se encuentran en su totalidad en la carpeta \emph{src}, mientras que sus secciones más relevantes se pueden leer en los apéndices de este informe.

\section{Consideraciones generales}
\subsection{Lenguaje de implementación}

Para implementar las soluciones algorítmicas desarrolladas en cada problema utilizamos el lenguaje C++, el cual presenta una serie de características muy convenientes. Este lenguaje es imperativo, al igual que el lenguaje de pseudocódigo utilizado para describir las soluciones y probar su correctitud. Adicionalmente, el mismo posee librerías estándar muy completas, versátiles y bien documentadas, lo cual permite abstraer el manejo de memoria, la implementación de estructuras de datos y algoritmos de uso frecuente, y provee mecanismos para realizar mediciones de tiempo de manera fidedigna.

\subsection{Algoritmo de ordenamiento}

En las implementaciones desarrollados para resolver los problemas planteados, utilizamos la función \emph{sort} de la Standard Template Library (STL). Para que la complejidad temporal de las soluciones se condiga con el análisis teórico, es necesario verificar que dicha función tenga efectivamente una complejidad temporal $O(n * log(n))$.

En la página de documentación oficial cppreference\footnote{http://en.cppreference.com/w/cpp/algorithm/sort}, se observa que a partir del standard C+11 de C++, la complejidad requerida para std::sort es de $O(n*log (n))$ 
comparaciones, y a lo sumo $O(n * log(n))$ swaps. Como utilizamos contenedores de acceso aleatorio en tiempo constante para desarrollar las soluciones, las operaciones de comparación y swap son $O(1)$. Por esto, efectivamente las rutinas de ordenamiento utilizadas tienen una complejidad de $O(n * log(n))$ operaciones.

% Con sólo esta informacion no podemos asegurar que el algoritmo en su totalidad tenga una complejidad temporal de $O(n*log (n))$  operaciones, por lo que buscamos que hace el algoritmo \emph{std::sort} revisando el código de \emph{algorithm.h}.

% esto no esta bien
 %Encontramos que, para casos con cantidad de elementos a ordenar menor a 64, hace un sort especial (el cual no nos interesa ya que queremos evaluar lo que pasa para $n$ grande). 

%esto depende exclusivamente de la implementacion que estás usando, y no del standard!
 %En casos mas grandes, realiza \emph{IntroSort}. \emph{IntroSort} intenta ordenar usando \emph{QuickSort}, si no lo resuelve en $n*log (n)$ pasos, usa HeapSort para garantizar $O(n*log (n))$ comparaciones.


\subsection{Mediciones de performance}
\label{consideraciones-mediciones}

Para llevar a cabo mediciones de performance sobre las implementaciones desarrolladas, medimos el tiempo consumido para resolver instancias de sucesivos tamaños en función de un parámetro a definir según el caso. Procuramos medir exclusivamente el tiempo consumido por la etapa de resolución, ignorando tareas adicionales propias al proceso como, por ejemplo, la generación de la instancia a ser resuelta.

La función del sistema que se escogió para medir intervalos de tiempo es la siguiente:

\begin{verbatim}
  int clock_gettime(clockid_t clk_id, struct timespec *tp);
\end{verbatim}

de la librería \emph{time.h}. La misma nos permite realizar mediciones de alta resolución, específicas al tiempo de ejecución del proceso que la invoca (y no al sistema en su totalidad), configurando el parámetro clk\_id con el valor CLOCK\_PROCESS\_CPUTIME\_ID\footnote{http://linux.die.net/man/3/clock\_gettime}.

Por otro lado, dado que la medición de tiempos en un sistema operativo activo introduce inherentemente un cierto nivel de ruido, cada medición se realizó múltiples veces. Una vez obtenidos los distintos valores para una misma medición (es decir, para diferentes instancias del mismo tamaño), registramos como valor definitivo la mediana de la serie de valores. Escogimos este criterio en vez de, por ejemplo, tomar la media, ya que utilizar la mediana es menos susceptible a la presencia de valores atípicos o \emph{outliers}.

\newpage


%%%%%%%%%%%%%%%%%%%%%%%%%%%%%%%%%%%%%%%%%%%%%%%%%%%%%%%%%%%%%%%%%%%%%%%%%%%%%%%
%% Problema 1: Robanúmeros			                              %%
%%%%%%%%%%%%%%%%%%%%%%%%%%%%%%%%%%%%%%%%%%%%%%%%%%%%%%%%%%%%%%%%%%%%%%%%%%%%%%%

\section{Problema 1: Robanúmeros}

\subsection{Descripción del problema}
\label{problema1-descripcion}
Dada una serie de pueblos con sus respectivas ubicaciones, se desea determinar un plan de construcción que permita abastecer a todos ellos con gas natural. Es posible construir una cantidad limitada de centrales distribuidoras de gas, cada una de ellas en un pueblo, así como tuberías uniendo las distintas ubicaciones. Un pueblo se considera abastecido de gas si existe un camino por medio de tuberías desde el mismo hasta alguna central distribuidora, incluso si dicho camino pasa por otros pueblos. Por ejemplo, si el pueblo $p_1$ tiene una central y los pueblos $p_2$ y $p_3$ no, se puede construir una tubería del pueblo $p_1$ al pueblo $p_2$ y una del pueblo $p_2$ al pueblo $p_3$, quedando abastecidos de esta forma los tres pueblos sin necesidad de realizar una tubería directa entre $p_1$ y $p_3$.

Adicionalmente, se define el riesgo asociado a un plan de construcción como la mayor longitud de todas las tuberías que se deben construir para realizar el mismo. Entre todos los posibles planes que abastecen a todos los pueblos, se desea hallar aquel que tenga un riesgo asociado mínimo (o cualquiera de ellos, si hay más de uno).

La ubicación de cada pueblo queda determinada por sus coordenadas $x$ e $y$, por lo cual si tenemos $n$ pueblos y podemos construir a lo sumo $k$ centrales, cada instancia del problema se puede codificar de la siguiente manera:

\begin{verbatim}
  n k
  x1 y1
  x2 y2
  ...
  xn yn
\end{verbatim}

Luego, lo que buscamos es distribuir a lo sumo $k$ centrales entre los $n$ pueblos y realizar las conexiones entre ellos que sean necesarias para que todos queden abastecidos, de forma tal que se minimice la máxima longitud entre todas las tuberías construidas. Para representar una solución, alcanza con determinar la cantidad $q$ de centrales construidas, la cantidad $m$ de tuberías construidas, el índice de los pueblos donde se construyeron las centrales $c_1, c_2, ... c_q$, y los índices $i$, $j$ de los pueblos conectados por cada tubería:

\begin{verbatim}
  q m
  c1 c2 ... cq
  i1 j1
  ...
  im jm
\end{verbatim}

\subsubsection{Ejemplos y observaciones}

A continuación incluimos un ejemplo donde se tienen $10$ pueblos y se pueden construir a lo sumo $3$ centrales.
\begin{verbatim}
  10 3
  1 1
  1 2
  2 1
  7 8
  9 10
  8 7
  25 26
  27 28
  28 28
  27 27
\end{verbatim}

Dado que los pueblos se encuentran agrupados en tres grupos marcados $(1, 2, 3)$, $(4, 5, 6)$ y $(7, 8, 9, 10)$, es razonable que una solución óptima como la presentada aquí debajo consista en colocar una central por grupo, evitando así tuberías largas que unan pueblos alejados:

\begin{verbatim}
  3 7
  1 4 7 
  5 4
  10 7
  6 4
  2 1
  3 1
  8 10
  9 8
\end{verbatim}

Destacamos como observación el hecho de que no es una prioridad la minimización de la cantidad de centrales ni la cantidad de tuberías construidas, y por lo tanto dada una solución óptima, cualquier otra con mayor cantidad de construcciones será equivalente mientras tenga un mismo riesgo asociado.

\subsection{Desarrollo de la solución}
\label{problema1-desarrollo}
Para resolver este problema diseñamos un algoritmo que utilize la técnica de programación dinámica. Lo primero que hicimos fue pensar la función recursiva que utilizará el algoritmo para detérminar cual es el movimiento óptimo para hacer. La signatura de la función será:

$opt(i,j) = c$

Donde $C$ es el total de los puntos que sumamos e $i$ y $j$ son los indices de las cartas que quedan en juego, $i$ es la primer carta que queda desde la izquierda, y $j$ es la primer carta desde la derecha.

\subsection{Función Recursiva:}

Pensamos la función según el siguiente enfoque: Al finalizar el juego, la suma de las cartas que robamos será la suma de todas las cartas menos la suma de las cartas que robo el rival, esto es así porque según las reglas del juego no podemos elegir no robar cartas:

$\sum{i=0}^{n} c_i = \sum{j \in A} c_j + \sum{k \in B} c_k$

Donde A es el conjunto de cartas que robó el jugador A y B es el conjunto de caartas que robó el jugador B.

Lo que pensamos es que maximizar la sumatoria de puntos que robemos nosotros es lo mismo que minimizar la suma de puntos robada por el otro jugador. Además sabemos que el otro jugador también jugara de forma óptima, por lo tanto llegamos a la conclusión de que la función recusrsiva para resolver este problema es la siguiente:

$opt(0,0) = 0$
$oppt(i,i) = c_i$
$opt(i,j) = \sum cartas - min(opt(i+1, j), ..., opt(j,j), opt(i, j-1), ... ,opt(i,i)$

Vale aclarar que por cómo pensamos la función, siempre pasa que $i \leq j$.

Los casos base ocurren cuando no queda ninguna carta, o cuando queda una sola carta. Al no quedar cartas el jugador no puede ganar ningún punto, y al quedar una sola carta, las reglas del juego indican que el jugador debe robarla.

Lo que dice el paso recursivo de esta función es que el jugador debe robar cartas de forma tal que minimize los puntos que pueda ganar el contrincante, sabiendo que este también jugará de forma óptima. Por ejemplo, si el valor mínimo de la función $opt$ es $opt(i+3,j)$ significa que el jugador debe robar las primeras 3 cartas empezando por la izuquierda.

\subsection{Implementación:}

Para implementar un algoritmo que utiliza la técnica de programación dinámica necesitamos de una estructura para guardar los resultados que vamos computando para no tener que recalcularlos, y de esta forma lograr una buena complejidad temporal.

Para resolver este algoritmo utilizamos una matriz de $n * n$ donde $n$ es la cantidad de cartas al inciar el juego. Durante este informe llamaremos a la matriz $msp$.

El valor de la casilla $msp[i][j]$ será por un lado la cantidad de puntos óptima que se podrá lograr a partir de las cartas que quedan ($c_i ... c_j$) y por otro lado dos valores que indican las cartas que quedarán luego de que el jugador que le toque realize su turno, estos valores se utilizan para que dada una secuencia de cartas se pueda realizar un seguimiento de todos los turnos que se realizen hasta terminar el juego.

Por ejemplo si en la casilla $msp[i][j]$ tiene los valores $i+k, j$ significa que en ese turno el jugador robo k cartas empezando desde la izquierda y la proxima casilla que se debe obserbar para continuar con el seguimiento del juego es la casilla $msp[i+k][j]$. Además, si los valores de la casilla son $-1, -1$ significa que en ese turno el jugador robó todas las cartas restantes, terminando de esta forma el juego.



\subsection{Complejidad temporal}
\label{problema1-complejidad}
Nosotros afirmamos que nuestro algoritmo tiene una complejidad de $O(n^3 * k)$ operaciones. Vamos a demostrar esto de a partes.

Antes de empezar, vamos a hacer unas definiciones de tipo, válidas para los pseudocódigos de esta sección.
\begin{pseudo}
\State Tipo de dato Casillero es Tupla $\langle$ fila : entero, columna : entero, k : entero $\rangle$
\State Tipo de dato Entrada es Tupla $\langle$ n : entero, k\_max : entero, tablero : vector $\langle$ vector $\langle$ entero $\rangle$ $\rangle$, origen : Casillero, destino : Casillero $\rangle$
\State Tipo de dato Salida es Tupla $\langle$ saltos : entero, caminoMinimo : list $\langle$ Casillero$\rangle$ $\rangle$
\end{pseudo}

Desde un punto de vista mas ``macro'', nuestro algoritmo hace lo siguiente:

\begin{pseudo}
    \Procedure{Saltos en la \emph{Matrix}}{}
        \State Entrada E $\leftarrow$ leerEntrada() \Ode{n^2}
        \State Salida S(V) $\leftarrow$ resolver(E) \Ode{n^3*k}
        \State Imprimir(S) \Ode{n}
        \State \textbf{return} 0
    \EndProcedure
\end{pseudo}

\textbf{leerEntrada()} se encarga de cargar los datos vía \emph{stdin}. Por esto, realiza una cantidad constante de asignaciones todas las cuales son $O(1)$ ya que los tipos de datos son enteros o Casilleros (los cuales se componen de tres enteros). Una vez cargados los datos, creamos el tablero. El tablero es vector $\langle$ vector $\langle$ entero $\rangle$ $\rangle$ que representa el campo de juego. Por esto, tiene tamaño $n*n$ y realiza justamente $n^2$ asignaciones de enteros. Sumando todo esto, \textbf{leerEntrada()} tiene un costo de $O(n^2)$.

Sin embargo, cabe aclarar que después realizamos una asignación (ya que la Entrada E no es pasada por referencia). Esta asignación de Entrada se compone de tres asignaciones de enteros ($O(1)$ cada una), un asignación de tablero ($O(n^2)$) y dos asignaciones de Casillero ($O(1)$ cada uno). Pero esto no cambia el costo total de $O(n^2)$.

\textbf{resolver(E)} es el núcleo de nuestro algoritmo y es lo que más cuesta, y por esta razón lo vamos analizar con más detalle.

\begin{pseudo}
\State Tipo de dato Matriz3D es vector $\langle$ vector $\langle$ vector $\langle$ Tupla $\langle$ Casillero, k $\rangle$ $\rangle$ $\rangle$ $\rangle$
    \Procedure{resolver}{$Entrada E$}
        \State Matriz3D yaPase $\leftarrow$ crearMatriz() \Ode{1}
		\Comment{Empieza BFS, pero modificado para nuestra conveniencia}
		\State q $\leftarrow$ crearCola() \Ode{1}
		\Comment{La funcion que sigue marca al casillero e.origen en yaPase con cantidad de saltos 0}
		\State marcoComo(e.origen, yaPase, 0) \Ode{1}
		
		\State encolar(e.origen, q) \Ode{1}
		
		\While{!q.empty()} \Ode{1}
			\State c $\leftarrow$ q.desencolar() \Ode{1}
			\If{c == destino}
				\State break
			\EndIf
			
			\For{cada Casillero u en losAdyacentes(e,c)} \Ode{n}
			    \If{!estaEn(u,yaPase)} \Ode{1}
					\State marcoComo(u, yaPase, cantSaltos(c) + 1) \Ode{1}
					\State encolar(u, q) \Ode{1}
				\EndIf
      		\EndFor
		\EndWhile

      	\Comment{Terminamos BFS y pasamos a escribir la salida. escriboSalida en nuestro codigo no es una funcion aparte, pero para este pseudocodigo lo decidimos dejar así.}
      	\State escriboSalida(S, yaPase) \Ode{n}
      	\State \textbf{return} S
        
    \EndProcedure
\end{pseudo}

La parte importante de resolver, es el BFS que realizamos, ya que las otras partes del algoritmo tienen complejidad menor.

Vamos a empezar diciendo que un casillero cualquiera puede tener a lo sumo $2*(n-1)$ casilleros adyacentes (toda su fila y toda su columna, menos él mismo). Por esto, cuando realizamos losAdyacentes, buscamos a lo sumo $O(n)$ adyacentes. Si ya sabemos que estamos fuera del rango del tablero, no vale la pena seguir buscando. Por esta misma razón, el ciclo for que utiliza los adyacentes corre a lo sumo $O(n)$ veces por cada casillero.

Teniendo en cuenta esto, vamos a correr el ciclo while grande a lo sumo una vez por casillero. Cabe recordar que tenemos $n*n*k$ casilleros en total: un casillero por cada celda del tablero, y ese tablero lo tenemos k veces. Lo peor que puede suceder es tener que correr el while para cada casillero, por lo que tendríamos un costo total de $O(n*n*n*k)$, que es igual a $O(n^3*k)$.

Al escribir la salida, escribimos la cantidad de saltos, y cuáles fueron dichos saltos, osea cuál fue el camino que se tomó desde el inicio hasta el dstino. Podemos observar que la cantidad de saltos es $O(n)$. En el peor caso, todos los casilleros tienen potencia 1 (lo menor que pueden valer) y el jugador no tiene potencias extras para gastar ($k=0$). En este caso, estamos en el primer ejemplo de la sección  \ref{problema3-descripcion}. En este caso, una solución óptima es recorrer desde la fila origen hasta la fila destino, y luego de la columna origen hasta la columna destino. Tanto las filas como las columnas pueden estar separadas a lo sumo por $n-1$ casilleros. Por esto, el camino más corto esta claramente acotado por $2*n$, por lo que tenemos a lo sumo $O(2*n)$ iteraciones de ciclo. Por todo esto, podemos afirmar que \textbf{Imprimir(S)} tiene una complejidad de $O(n)$

Finalmente, \textbf{Imprimir(S)} es la función encargada de imprimir el resultado por pantalla. Vamos a utilizar un pseudocodigo para probar su complejidad.

\begin{pseudo}
    \Procedure{Imprimir}{$Salida s$}
        \State imprimir(s.saltos) \Ode{1}

		\For{cada Casillero c en s.caminoMinimo} \Ode{1}
        \State imprimir(c.fila) \Ode{1}
        \State imprimir(c.columna) \Ode{1}
        \State imprimir(kUsado(c, padre(c))) \Ode{1}
        \Comment{Este kUsado en nuestro algoritmo en C++ lo obtenemos sabiendo cual era nuestro padre con un iterador auxiliar. Como esto es pseudocodigo decidimos dejarlo de esta manera}
      	\EndFor
        
    \EndProcedure
\end{pseudo}

Como se puede observar, todas las impresiones por pantalla son $O(1)$ y la guarda del for también cuesta $O(1)$. Sin embargo, falta justificar cuantas veces corre el ciclo for. Como probamos anteriormente, el camino está acotado por $2*n$. Por esto, realizamos $O(n)$ cantidad de ciclos for.

Sumando todas las complejidades, llegamos a obtener una complejidad temporal total de $O(n^3 * k)$, con lo que cumplimos con la complejidad temporal pedida.

\label{problema1-demostracion}
\subsection{Demostración de correctitud}
Como dijimos anteriormente lo que se busca en encontrar un camino de mínimas aristas entre el casillero de entrada con $k_{max}$ osea el estado $(x_e,y_e,k_{max})$ hasta la salida sin importar las unidades de potencia restante, osea $(x_s,y_s,i)$ para algún i.

Como sabemos si realizamos BFS desde un nodo v en el grafo G, para cada nodo u $\in$ G, encuentra el camino de mínimas aristas entre v y u. Por eso al aplicar BFS desde $(x_e,y_e,k_{max})$ encontramos el camino de mínimas aristas entre él y la salida. Lo único que queda hacer es elegir entre los caminos mínimos de los estados  $(x_s,y_s,i)$ cual es el que tiene menos aristas y esa es la solución. 

\subsection{Experimentación}
\label{problema1-experimentacion}
\emph{Nota:} Como vamos a hablar mucho de las distintas podas vamos a denominarlas de la siguiente manera:
\begin{itemize}
\item Poda 1: Si aunque pongamos piezas en los casilleros que me quedan revisar, no llegamos a alcanzar la cantidad de piezas de la solución óptima encontrada a ese momento, corto.
\item Poda 2: Similar a la poda 1, pero además revisamos hasta $m$ casillas para adelante teniendo en cuenta las restricciones ya existentes en el tablero.
\end{itemize}

En este ejercicio, vamos a enfocar la experimentación orientándolo a la comparación entre las distintas podas, así como también comparar las mismas contra no usar ninguna poda. Cabe aclarar que no usar ninguna poda tiene un método de corte \emph{naïf}: cortar si logré llenar el tablero.

Primero probamos todos los casos posibles para n = 1 a n = 3 sin poda y para las dos podas que propusimos.

En los siguientes gráficos podemos observar que utilizar podas mejora drásticamente el tiempo de ejecución, respecto a no utilizar podas. No logramos realizar todos los casos de 3*3 sin podas ya que necesitaba un tiempo muy grande.

\begin{center}
  \begin{figure}[H]
    \includegraphics[width=0.5\linewidth]{problema3/graficos/comparacion_podas_todas_instancias_llamados.eps}
    \includegraphics[width=0.5\linewidth]{problema3/graficos/comparacion_podas_todas_instancias_tiempo.eps}
    \caption{Comparación con y sin podas. Izquierda: llamados recursivos, Derecha: tiempo}
    \label{fig:problema3-sin-y-con-podas}
  \end{figure}
\end{center}

En estos gráficos podemos observar que utilizar podas mejora tanto en la cantidad de llamados recursivos, así como también mejora en tiempo de ejecución. A su vez, si bien la cantidad de llamados recursivos que realiza el algoritmo al utilizar la poda 2 es menor a la cantidad de llamados recursivos que realiza al utilizar la poda 1, como el n es chico la cantidad de casos que poda no logra contrarestar el overhead que genera por lo que el tiempo de ejecución del algoritmo utilizando poda 2 es claramente mayor a utilizar la poda 1.

No usar podas es tan malo porque revisa todos los tableros posibles, y recién una vez revisados todos termina (salvando el caso en que justo encuentre el tablero completo). Esto implica que ya para tableros de 3*3 es un número grande de tableros posibles, y para tableros aún más grandes el algoritmo tarda una cantidad incomparable de tiempo con respecto a las podas. Por esta razón, sin podas no esta presente en el tercer gráfico de arriba, ni en los gráficos de abajo.

A continuacion experimentamos con 50 casos de 4*4 generados pseudo-aleatoriamente pero con una semilla constante. Cabe aclarar que se comparan los resultados de las dos podas para la misma entrada en cada caso.

\begin{center}
  \begin{figure}[H]
    \includegraphics[width=\linewidth]{problema3/graficos/comparacion_podas1y2_llamados_recursivos.eps}
    \caption{Comparación de podas, según llamados recursivos}
    \label{fig:problema3-podas-llamados}
  \end{figure}
\end{center}

\begin{center}
  \begin{figure}[H]
\includegraphics[width=\linewidth]{problema3/graficos/comparacion_podas1y2_tiempo.eps}
    \caption{Comparación de podas, según tiempo de ejecución}
    \label{fig:problema3-podas-tiempo}
  \end{figure}
\end{center}


En el gráfico \ref{fig:problema3-podas-llamados}, podemos observar claramente que la poda 2 logra podar mas casos y por consiguiente, realiza menos llamados recursivos. Vemos que al aumentar el n, la cantidad de casos que logra podar aumenta, logrando en la mayoria de los casos superar el overhead como se ve en la figura \ref{fig:problema3-podas-tiempo}. Sin embargo cabe aclarar que no siempre logra compensar lo suficiente el costo adicional de $O(m)$.

Como ejemplo, el caso 35 logra podar lo suficiente para lograr 4 veces menos llamados recursivos. Sin embargo, logra solamente ser un 25\% (aproximadamente) mejor en tiempo de ejecución.

Por otro lado, la poda 1 también tiene beneficios. Es considerablemente más fácil de implementar que la poda 2. A su vez, la poda 1 no depende de la cantidad de colores existentes, mientras que la poda 2 sí lo hace. La poda 2 crea un $vector<int>$ donde la cantidad de posiciones se corresponde con la cantidad de colores existentes. Si bien no es un problema muy grave, puede ocurrir que sí lo sea si tenemos una computadora con memoria acotada y tenemos un caso donde la cantidad de colores es muy alta, como por ejemplo un millón de colores.

Curiosamente, si bien utilizar la poda 1 incurre en una menor cantidad de tiempo de ejecución para casos mas chicos (como por ejemplo tableros de tamaño de 3*3), esto no se mantiene para casos más grandes (como tableros de tamaño 4*4). Es decir, la poda 2 comienza a tener más importancia cuando el tablero es más grande. Creemos que esto se debe a que las podas que realiza la poda 2 tienen más impacto con respecto al costo de realizar dicha poda en tableros de tamaño más grande.

\subsection{Conclusión}
\label{problema1-resultados}
Para concluir, nos parece importante destacar que la complejidad temporal de nuestra solución es dominada por la etapa de ordenamiento, lo cual provee una cota inferior a la complejidad temporal de nuestra solución, debido a que los algoritmos de ordenamiento por comparaciones son $\Omega(n * log(n))$. Sin embargo, en el caso en que los datos de entrada se encuentren ordenados o dentro de un rango acotado\footnote{Considerar casos donde la entrada solo admite días dentro de un rango es razonable dado el dominio del problema.}, la instancia se puede resolver en tiempo $O(n)$.

%  esto lo comento porque la etapa de lectura de datos no la incluímos en el análisis teórico, y no estoy seguro de que es imposible resolver el problema en tiempo mejor que O(n) para entradas ordenadas.
% Por otro lado, sin contar la etapa de ordenamiento, realizamos a lo sumo $O(n)$ operaciones, ya que podríamos tener que recorrer el vector por lo menos dos veces: una para guardar los datos y otra para saber cuál es el día óptimo para colocar al inspector. Por esto, podemos asegurar que por más que los datos vengan ordenados no vamos a poder mejorar la complejidad temporal de $O(n)$.

\newpage


%%%%%%%%%%%%%%%%%%%%%%%%%%%%%%%%%%%%%%%%%%%%%%%%%%%%%%%%%%%%%%%%%%%%%%%%%%%%%%%
%% Problema 2: La centralita (de gas)		                              %%
%%%%%%%%%%%%%%%%%%%%%%%%%%%%%%%%%%%%%%%%%%%%%%%%%%%%%%%%%%%%%%%%%%%%%%%%%%%%%%%

\section{Problema 2: La centralita (de gas)}

\subsection{Descripción del problema}
\label{problema2-descripcion}
Dada una serie de pueblos con sus respectivas ubicaciones, se desea determinar un plan de construcción que permita abastecer a todos ellos con gas natural. Es posible construir una cantidad limitada de centrales distribuidoras de gas, cada una de ellas en un pueblo, así como tuberías uniendo las distintas ubicaciones. Un pueblo se considera abastecido de gas si existe un camino por medio de tuberías desde el mismo hasta alguna central distribuidora, incluso si dicho camino pasa por otros pueblos. Por ejemplo, si el pueblo $p_1$ tiene una central y los pueblos $p_2$ y $p_3$ no, se puede construir una tubería del pueblo $p_1$ al pueblo $p_2$ y una del pueblo $p_2$ al pueblo $p_3$, quedando abastecidos de esta forma los tres pueblos sin necesidad de realizar una tubería directa entre $p_1$ y $p_3$.

Adicionalmente, se define el riesgo asociado a un plan de construcción como la mayor longitud de todas las tuberías que se deben construir para realizar el mismo. Entre todos los posibles planes que abastecen a todos los pueblos, se desea hallar aquel que tenga un riesgo asociado mínimo (o cualquiera de ellos, si hay más de uno).

La ubicación de cada pueblo queda determinada por sus coordenadas $x$ e $y$, por lo cual si tenemos $n$ pueblos y podemos construir a lo sumo $k$ centrales, cada instancia del problema se puede codificar de la siguiente manera:

\begin{verbatim}
  n k
  x1 y1
  x2 y2
  ...
  xn yn
\end{verbatim}

Luego, lo que buscamos es distribuir a lo sumo $k$ centrales entre los $n$ pueblos y realizar las conexiones entre ellos que sean necesarias para que todos queden abastecidos, de forma tal que se minimice la máxima longitud entre todas las tuberías construidas. Para representar una solución, alcanza con determinar la cantidad $q$ de centrales construidas, la cantidad $m$ de tuberías construidas, el índice de los pueblos donde se construyeron las centrales $c_1, c_2, ... c_q$, y los índices $i$, $j$ de los pueblos conectados por cada tubería:

\begin{verbatim}
  q m
  c1 c2 ... cq
  i1 j1
  ...
  im jm
\end{verbatim}

\subsubsection{Ejemplos y observaciones}

A continuación incluimos un ejemplo donde se tienen $10$ pueblos y se pueden construir a lo sumo $3$ centrales.
\begin{verbatim}
  10 3
  1 1
  1 2
  2 1
  7 8
  9 10
  8 7
  25 26
  27 28
  28 28
  27 27
\end{verbatim}

Dado que los pueblos se encuentran agrupados en tres grupos marcados $(1, 2, 3)$, $(4, 5, 6)$ y $(7, 8, 9, 10)$, es razonable que una solución óptima como la presentada aquí debajo consista en colocar una central por grupo, evitando así tuberías largas que unan pueblos alejados:

\begin{verbatim}
  3 7
  1 4 7 
  5 4
  10 7
  6 4
  2 1
  3 1
  8 10
  9 8
\end{verbatim}

Destacamos como observación el hecho de que no es una prioridad la minimización de la cantidad de centrales ni la cantidad de tuberías construidas, y por lo tanto dada una solución óptima, cualquier otra con mayor cantidad de construcciones será equivalente mientras tenga un mismo riesgo asociado.

\subsection{Desarrollo de la solución}
\label{problema2-desarrollo}
Para resolver este problema diseñamos un algoritmo que utilize la técnica de programación dinámica. Lo primero que hicimos fue pensar la función recursiva que utilizará el algoritmo para detérminar cual es el movimiento óptimo para hacer. La signatura de la función será:

$opt(i,j) = c$

Donde $C$ es el total de los puntos que sumamos e $i$ y $j$ son los indices de las cartas que quedan en juego, $i$ es la primer carta que queda desde la izquierda, y $j$ es la primer carta desde la derecha.

\subsection{Función Recursiva:}

Pensamos la función según el siguiente enfoque: Al finalizar el juego, la suma de las cartas que robamos será la suma de todas las cartas menos la suma de las cartas que robo el rival, esto es así porque según las reglas del juego no podemos elegir no robar cartas:

$\sum{i=0}^{n} c_i = \sum{j \in A} c_j + \sum{k \in B} c_k$

Donde A es el conjunto de cartas que robó el jugador A y B es el conjunto de caartas que robó el jugador B.

Lo que pensamos es que maximizar la sumatoria de puntos que robemos nosotros es lo mismo que minimizar la suma de puntos robada por el otro jugador. Además sabemos que el otro jugador también jugara de forma óptima, por lo tanto llegamos a la conclusión de que la función recusrsiva para resolver este problema es la siguiente:

$opt(0,0) = 0$
$oppt(i,i) = c_i$
$opt(i,j) = \sum cartas - min(opt(i+1, j), ..., opt(j,j), opt(i, j-1), ... ,opt(i,i)$

Vale aclarar que por cómo pensamos la función, siempre pasa que $i \leq j$.

Los casos base ocurren cuando no queda ninguna carta, o cuando queda una sola carta. Al no quedar cartas el jugador no puede ganar ningún punto, y al quedar una sola carta, las reglas del juego indican que el jugador debe robarla.

Lo que dice el paso recursivo de esta función es que el jugador debe robar cartas de forma tal que minimize los puntos que pueda ganar el contrincante, sabiendo que este también jugará de forma óptima. Por ejemplo, si el valor mínimo de la función $opt$ es $opt(i+3,j)$ significa que el jugador debe robar las primeras 3 cartas empezando por la izuquierda.

\subsection{Implementación:}

Para implementar un algoritmo que utiliza la técnica de programación dinámica necesitamos de una estructura para guardar los resultados que vamos computando para no tener que recalcularlos, y de esta forma lograr una buena complejidad temporal.

Para resolver este algoritmo utilizamos una matriz de $n * n$ donde $n$ es la cantidad de cartas al inciar el juego. Durante este informe llamaremos a la matriz $msp$.

El valor de la casilla $msp[i][j]$ será por un lado la cantidad de puntos óptima que se podrá lograr a partir de las cartas que quedan ($c_i ... c_j$) y por otro lado dos valores que indican las cartas que quedarán luego de que el jugador que le toque realize su turno, estos valores se utilizan para que dada una secuencia de cartas se pueda realizar un seguimiento de todos los turnos que se realizen hasta terminar el juego.

Por ejemplo si en la casilla $msp[i][j]$ tiene los valores $i+k, j$ significa que en ese turno el jugador robo k cartas empezando desde la izquierda y la proxima casilla que se debe obserbar para continuar con el seguimiento del juego es la casilla $msp[i+k][j]$. Además, si los valores de la casilla son $-1, -1$ significa que en ese turno el jugador robó todas las cartas restantes, terminando de esta forma el juego.



\subsection{Complejidad temporal}
\label{problema2-complejidad}
Nosotros afirmamos que nuestro algoritmo tiene una complejidad de $O(n^3 * k)$ operaciones. Vamos a demostrar esto de a partes.

Antes de empezar, vamos a hacer unas definiciones de tipo, válidas para los pseudocódigos de esta sección.
\begin{pseudo}
\State Tipo de dato Casillero es Tupla $\langle$ fila : entero, columna : entero, k : entero $\rangle$
\State Tipo de dato Entrada es Tupla $\langle$ n : entero, k\_max : entero, tablero : vector $\langle$ vector $\langle$ entero $\rangle$ $\rangle$, origen : Casillero, destino : Casillero $\rangle$
\State Tipo de dato Salida es Tupla $\langle$ saltos : entero, caminoMinimo : list $\langle$ Casillero$\rangle$ $\rangle$
\end{pseudo}

Desde un punto de vista mas ``macro'', nuestro algoritmo hace lo siguiente:

\begin{pseudo}
    \Procedure{Saltos en la \emph{Matrix}}{}
        \State Entrada E $\leftarrow$ leerEntrada() \Ode{n^2}
        \State Salida S(V) $\leftarrow$ resolver(E) \Ode{n^3*k}
        \State Imprimir(S) \Ode{n}
        \State \textbf{return} 0
    \EndProcedure
\end{pseudo}

\textbf{leerEntrada()} se encarga de cargar los datos vía \emph{stdin}. Por esto, realiza una cantidad constante de asignaciones todas las cuales son $O(1)$ ya que los tipos de datos son enteros o Casilleros (los cuales se componen de tres enteros). Una vez cargados los datos, creamos el tablero. El tablero es vector $\langle$ vector $\langle$ entero $\rangle$ $\rangle$ que representa el campo de juego. Por esto, tiene tamaño $n*n$ y realiza justamente $n^2$ asignaciones de enteros. Sumando todo esto, \textbf{leerEntrada()} tiene un costo de $O(n^2)$.

Sin embargo, cabe aclarar que después realizamos una asignación (ya que la Entrada E no es pasada por referencia). Esta asignación de Entrada se compone de tres asignaciones de enteros ($O(1)$ cada una), un asignación de tablero ($O(n^2)$) y dos asignaciones de Casillero ($O(1)$ cada uno). Pero esto no cambia el costo total de $O(n^2)$.

\textbf{resolver(E)} es el núcleo de nuestro algoritmo y es lo que más cuesta, y por esta razón lo vamos analizar con más detalle.

\begin{pseudo}
\State Tipo de dato Matriz3D es vector $\langle$ vector $\langle$ vector $\langle$ Tupla $\langle$ Casillero, k $\rangle$ $\rangle$ $\rangle$ $\rangle$
    \Procedure{resolver}{$Entrada E$}
        \State Matriz3D yaPase $\leftarrow$ crearMatriz() \Ode{1}
		\Comment{Empieza BFS, pero modificado para nuestra conveniencia}
		\State q $\leftarrow$ crearCola() \Ode{1}
		\Comment{La funcion que sigue marca al casillero e.origen en yaPase con cantidad de saltos 0}
		\State marcoComo(e.origen, yaPase, 0) \Ode{1}
		
		\State encolar(e.origen, q) \Ode{1}
		
		\While{!q.empty()} \Ode{1}
			\State c $\leftarrow$ q.desencolar() \Ode{1}
			\If{c == destino}
				\State break
			\EndIf
			
			\For{cada Casillero u en losAdyacentes(e,c)} \Ode{n}
			    \If{!estaEn(u,yaPase)} \Ode{1}
					\State marcoComo(u, yaPase, cantSaltos(c) + 1) \Ode{1}
					\State encolar(u, q) \Ode{1}
				\EndIf
      		\EndFor
		\EndWhile

      	\Comment{Terminamos BFS y pasamos a escribir la salida. escriboSalida en nuestro codigo no es una funcion aparte, pero para este pseudocodigo lo decidimos dejar así.}
      	\State escriboSalida(S, yaPase) \Ode{n}
      	\State \textbf{return} S
        
    \EndProcedure
\end{pseudo}

La parte importante de resolver, es el BFS que realizamos, ya que las otras partes del algoritmo tienen complejidad menor.

Vamos a empezar diciendo que un casillero cualquiera puede tener a lo sumo $2*(n-1)$ casilleros adyacentes (toda su fila y toda su columna, menos él mismo). Por esto, cuando realizamos losAdyacentes, buscamos a lo sumo $O(n)$ adyacentes. Si ya sabemos que estamos fuera del rango del tablero, no vale la pena seguir buscando. Por esta misma razón, el ciclo for que utiliza los adyacentes corre a lo sumo $O(n)$ veces por cada casillero.

Teniendo en cuenta esto, vamos a correr el ciclo while grande a lo sumo una vez por casillero. Cabe recordar que tenemos $n*n*k$ casilleros en total: un casillero por cada celda del tablero, y ese tablero lo tenemos k veces. Lo peor que puede suceder es tener que correr el while para cada casillero, por lo que tendríamos un costo total de $O(n*n*n*k)$, que es igual a $O(n^3*k)$.

Al escribir la salida, escribimos la cantidad de saltos, y cuáles fueron dichos saltos, osea cuál fue el camino que se tomó desde el inicio hasta el dstino. Podemos observar que la cantidad de saltos es $O(n)$. En el peor caso, todos los casilleros tienen potencia 1 (lo menor que pueden valer) y el jugador no tiene potencias extras para gastar ($k=0$). En este caso, estamos en el primer ejemplo de la sección  \ref{problema3-descripcion}. En este caso, una solución óptima es recorrer desde la fila origen hasta la fila destino, y luego de la columna origen hasta la columna destino. Tanto las filas como las columnas pueden estar separadas a lo sumo por $n-1$ casilleros. Por esto, el camino más corto esta claramente acotado por $2*n$, por lo que tenemos a lo sumo $O(2*n)$ iteraciones de ciclo. Por todo esto, podemos afirmar que \textbf{Imprimir(S)} tiene una complejidad de $O(n)$

Finalmente, \textbf{Imprimir(S)} es la función encargada de imprimir el resultado por pantalla. Vamos a utilizar un pseudocodigo para probar su complejidad.

\begin{pseudo}
    \Procedure{Imprimir}{$Salida s$}
        \State imprimir(s.saltos) \Ode{1}

		\For{cada Casillero c en s.caminoMinimo} \Ode{1}
        \State imprimir(c.fila) \Ode{1}
        \State imprimir(c.columna) \Ode{1}
        \State imprimir(kUsado(c, padre(c))) \Ode{1}
        \Comment{Este kUsado en nuestro algoritmo en C++ lo obtenemos sabiendo cual era nuestro padre con un iterador auxiliar. Como esto es pseudocodigo decidimos dejarlo de esta manera}
      	\EndFor
        
    \EndProcedure
\end{pseudo}

Como se puede observar, todas las impresiones por pantalla son $O(1)$ y la guarda del for también cuesta $O(1)$. Sin embargo, falta justificar cuantas veces corre el ciclo for. Como probamos anteriormente, el camino está acotado por $2*n$. Por esto, realizamos $O(n)$ cantidad de ciclos for.

Sumando todas las complejidades, llegamos a obtener una complejidad temporal total de $O(n^3 * k)$, con lo que cumplimos con la complejidad temporal pedida.

\label{problema2-demostracion}
\subsection{Demostración de correctitud}
Como dijimos anteriormente lo que se busca en encontrar un camino de mínimas aristas entre el casillero de entrada con $k_{max}$ osea el estado $(x_e,y_e,k_{max})$ hasta la salida sin importar las unidades de potencia restante, osea $(x_s,y_s,i)$ para algún i.

Como sabemos si realizamos BFS desde un nodo v en el grafo G, para cada nodo u $\in$ G, encuentra el camino de mínimas aristas entre v y u. Por eso al aplicar BFS desde $(x_e,y_e,k_{max})$ encontramos el camino de mínimas aristas entre él y la salida. Lo único que queda hacer es elegir entre los caminos mínimos de los estados  $(x_s,y_s,i)$ cual es el que tiene menos aristas y esa es la solución. 

\subsection{Experimentación}
\label{problema2-experimentacion}
\emph{Nota:} Como vamos a hablar mucho de las distintas podas vamos a denominarlas de la siguiente manera:
\begin{itemize}
\item Poda 1: Si aunque pongamos piezas en los casilleros que me quedan revisar, no llegamos a alcanzar la cantidad de piezas de la solución óptima encontrada a ese momento, corto.
\item Poda 2: Similar a la poda 1, pero además revisamos hasta $m$ casillas para adelante teniendo en cuenta las restricciones ya existentes en el tablero.
\end{itemize}

En este ejercicio, vamos a enfocar la experimentación orientándolo a la comparación entre las distintas podas, así como también comparar las mismas contra no usar ninguna poda. Cabe aclarar que no usar ninguna poda tiene un método de corte \emph{naïf}: cortar si logré llenar el tablero.

Primero probamos todos los casos posibles para n = 1 a n = 3 sin poda y para las dos podas que propusimos.

En los siguientes gráficos podemos observar que utilizar podas mejora drásticamente el tiempo de ejecución, respecto a no utilizar podas. No logramos realizar todos los casos de 3*3 sin podas ya que necesitaba un tiempo muy grande.

\begin{center}
  \begin{figure}[H]
    \includegraphics[width=0.5\linewidth]{problema3/graficos/comparacion_podas_todas_instancias_llamados.eps}
    \includegraphics[width=0.5\linewidth]{problema3/graficos/comparacion_podas_todas_instancias_tiempo.eps}
    \caption{Comparación con y sin podas. Izquierda: llamados recursivos, Derecha: tiempo}
    \label{fig:problema3-sin-y-con-podas}
  \end{figure}
\end{center}

En estos gráficos podemos observar que utilizar podas mejora tanto en la cantidad de llamados recursivos, así como también mejora en tiempo de ejecución. A su vez, si bien la cantidad de llamados recursivos que realiza el algoritmo al utilizar la poda 2 es menor a la cantidad de llamados recursivos que realiza al utilizar la poda 1, como el n es chico la cantidad de casos que poda no logra contrarestar el overhead que genera por lo que el tiempo de ejecución del algoritmo utilizando poda 2 es claramente mayor a utilizar la poda 1.

No usar podas es tan malo porque revisa todos los tableros posibles, y recién una vez revisados todos termina (salvando el caso en que justo encuentre el tablero completo). Esto implica que ya para tableros de 3*3 es un número grande de tableros posibles, y para tableros aún más grandes el algoritmo tarda una cantidad incomparable de tiempo con respecto a las podas. Por esta razón, sin podas no esta presente en el tercer gráfico de arriba, ni en los gráficos de abajo.

A continuacion experimentamos con 50 casos de 4*4 generados pseudo-aleatoriamente pero con una semilla constante. Cabe aclarar que se comparan los resultados de las dos podas para la misma entrada en cada caso.

\begin{center}
  \begin{figure}[H]
    \includegraphics[width=\linewidth]{problema3/graficos/comparacion_podas1y2_llamados_recursivos.eps}
    \caption{Comparación de podas, según llamados recursivos}
    \label{fig:problema3-podas-llamados}
  \end{figure}
\end{center}

\begin{center}
  \begin{figure}[H]
\includegraphics[width=\linewidth]{problema3/graficos/comparacion_podas1y2_tiempo.eps}
    \caption{Comparación de podas, según tiempo de ejecución}
    \label{fig:problema3-podas-tiempo}
  \end{figure}
\end{center}


En el gráfico \ref{fig:problema3-podas-llamados}, podemos observar claramente que la poda 2 logra podar mas casos y por consiguiente, realiza menos llamados recursivos. Vemos que al aumentar el n, la cantidad de casos que logra podar aumenta, logrando en la mayoria de los casos superar el overhead como se ve en la figura \ref{fig:problema3-podas-tiempo}. Sin embargo cabe aclarar que no siempre logra compensar lo suficiente el costo adicional de $O(m)$.

Como ejemplo, el caso 35 logra podar lo suficiente para lograr 4 veces menos llamados recursivos. Sin embargo, logra solamente ser un 25\% (aproximadamente) mejor en tiempo de ejecución.

Por otro lado, la poda 1 también tiene beneficios. Es considerablemente más fácil de implementar que la poda 2. A su vez, la poda 1 no depende de la cantidad de colores existentes, mientras que la poda 2 sí lo hace. La poda 2 crea un $vector<int>$ donde la cantidad de posiciones se corresponde con la cantidad de colores existentes. Si bien no es un problema muy grave, puede ocurrir que sí lo sea si tenemos una computadora con memoria acotada y tenemos un caso donde la cantidad de colores es muy alta, como por ejemplo un millón de colores.

Curiosamente, si bien utilizar la poda 1 incurre en una menor cantidad de tiempo de ejecución para casos mas chicos (como por ejemplo tableros de tamaño de 3*3), esto no se mantiene para casos más grandes (como tableros de tamaño 4*4). Es decir, la poda 2 comienza a tener más importancia cuando el tablero es más grande. Creemos que esto se debe a que las podas que realiza la poda 2 tienen más impacto con respecto al costo de realizar dicha poda en tableros de tamaño más grande.

\subsection{Conclusión}
\label{problema2-resultados}
Para concluir, nos parece importante destacar que la complejidad temporal de nuestra solución es dominada por la etapa de ordenamiento, lo cual provee una cota inferior a la complejidad temporal de nuestra solución, debido a que los algoritmos de ordenamiento por comparaciones son $\Omega(n * log(n))$. Sin embargo, en el caso en que los datos de entrada se encuentren ordenados o dentro de un rango acotado\footnote{Considerar casos donde la entrada solo admite días dentro de un rango es razonable dado el dominio del problema.}, la instancia se puede resolver en tiempo $O(n)$.

%  esto lo comento porque la etapa de lectura de datos no la incluímos en el análisis teórico, y no estoy seguro de que es imposible resolver el problema en tiempo mejor que O(n) para entradas ordenadas.
% Por otro lado, sin contar la etapa de ordenamiento, realizamos a lo sumo $O(n)$ operaciones, ya que podríamos tener que recorrer el vector por lo menos dos veces: una para guardar los datos y otra para saber cuál es el día óptimo para colocar al inspector. Por esto, podemos asegurar que por más que los datos vengan ordenados no vamos a poder mejorar la complejidad temporal de $O(n)$.

\newpage


%%%%%%%%%%%%%%%%%%%%%%%%%%%%%%%%%%%%%%%%%%%%%%%%%%%%%%%%%%%%%%%%%%%%%%%%%%%%%%%
%% Problema 3: Saltos en La Matrix                                           %%
%%%%%%%%%%%%%%%%%%%%%%%%%%%%%%%%%%%%%%%%%%%%%%%%%%%%%%%%%%%%%%%%%%%%%%%%%%%%%%%

\section{Problema 3: Saltos en la Matrix}

\subsection{Descripción del problema}
\label{problema3-descripcion}
Dada una serie de pueblos con sus respectivas ubicaciones, se desea determinar un plan de construcción que permita abastecer a todos ellos con gas natural. Es posible construir una cantidad limitada de centrales distribuidoras de gas, cada una de ellas en un pueblo, así como tuberías uniendo las distintas ubicaciones. Un pueblo se considera abastecido de gas si existe un camino por medio de tuberías desde el mismo hasta alguna central distribuidora, incluso si dicho camino pasa por otros pueblos. Por ejemplo, si el pueblo $p_1$ tiene una central y los pueblos $p_2$ y $p_3$ no, se puede construir una tubería del pueblo $p_1$ al pueblo $p_2$ y una del pueblo $p_2$ al pueblo $p_3$, quedando abastecidos de esta forma los tres pueblos sin necesidad de realizar una tubería directa entre $p_1$ y $p_3$.

Adicionalmente, se define el riesgo asociado a un plan de construcción como la mayor longitud de todas las tuberías que se deben construir para realizar el mismo. Entre todos los posibles planes que abastecen a todos los pueblos, se desea hallar aquel que tenga un riesgo asociado mínimo (o cualquiera de ellos, si hay más de uno).

La ubicación de cada pueblo queda determinada por sus coordenadas $x$ e $y$, por lo cual si tenemos $n$ pueblos y podemos construir a lo sumo $k$ centrales, cada instancia del problema se puede codificar de la siguiente manera:

\begin{verbatim}
  n k
  x1 y1
  x2 y2
  ...
  xn yn
\end{verbatim}

Luego, lo que buscamos es distribuir a lo sumo $k$ centrales entre los $n$ pueblos y realizar las conexiones entre ellos que sean necesarias para que todos queden abastecidos, de forma tal que se minimice la máxima longitud entre todas las tuberías construidas. Para representar una solución, alcanza con determinar la cantidad $q$ de centrales construidas, la cantidad $m$ de tuberías construidas, el índice de los pueblos donde se construyeron las centrales $c_1, c_2, ... c_q$, y los índices $i$, $j$ de los pueblos conectados por cada tubería:

\begin{verbatim}
  q m
  c1 c2 ... cq
  i1 j1
  ...
  im jm
\end{verbatim}

\subsubsection{Ejemplos y observaciones}

A continuación incluimos un ejemplo donde se tienen $10$ pueblos y se pueden construir a lo sumo $3$ centrales.
\begin{verbatim}
  10 3
  1 1
  1 2
  2 1
  7 8
  9 10
  8 7
  25 26
  27 28
  28 28
  27 27
\end{verbatim}

Dado que los pueblos se encuentran agrupados en tres grupos marcados $(1, 2, 3)$, $(4, 5, 6)$ y $(7, 8, 9, 10)$, es razonable que una solución óptima como la presentada aquí debajo consista en colocar una central por grupo, evitando así tuberías largas que unan pueblos alejados:

\begin{verbatim}
  3 7
  1 4 7 
  5 4
  10 7
  6 4
  2 1
  3 1
  8 10
  9 8
\end{verbatim}

Destacamos como observación el hecho de que no es una prioridad la minimización de la cantidad de centrales ni la cantidad de tuberías construidas, y por lo tanto dada una solución óptima, cualquier otra con mayor cantidad de construcciones será equivalente mientras tenga un mismo riesgo asociado.

\subsection{Desarrollo de la solución}
\label{problema3-desarrollo}
Para resolver este problema diseñamos un algoritmo que utilize la técnica de programación dinámica. Lo primero que hicimos fue pensar la función recursiva que utilizará el algoritmo para detérminar cual es el movimiento óptimo para hacer. La signatura de la función será:

$opt(i,j) = c$

Donde $C$ es el total de los puntos que sumamos e $i$ y $j$ son los indices de las cartas que quedan en juego, $i$ es la primer carta que queda desde la izquierda, y $j$ es la primer carta desde la derecha.

\subsection{Función Recursiva:}

Pensamos la función según el siguiente enfoque: Al finalizar el juego, la suma de las cartas que robamos será la suma de todas las cartas menos la suma de las cartas que robo el rival, esto es así porque según las reglas del juego no podemos elegir no robar cartas:

$\sum{i=0}^{n} c_i = \sum{j \in A} c_j + \sum{k \in B} c_k$

Donde A es el conjunto de cartas que robó el jugador A y B es el conjunto de caartas que robó el jugador B.

Lo que pensamos es que maximizar la sumatoria de puntos que robemos nosotros es lo mismo que minimizar la suma de puntos robada por el otro jugador. Además sabemos que el otro jugador también jugara de forma óptima, por lo tanto llegamos a la conclusión de que la función recusrsiva para resolver este problema es la siguiente:

$opt(0,0) = 0$
$oppt(i,i) = c_i$
$opt(i,j) = \sum cartas - min(opt(i+1, j), ..., opt(j,j), opt(i, j-1), ... ,opt(i,i)$

Vale aclarar que por cómo pensamos la función, siempre pasa que $i \leq j$.

Los casos base ocurren cuando no queda ninguna carta, o cuando queda una sola carta. Al no quedar cartas el jugador no puede ganar ningún punto, y al quedar una sola carta, las reglas del juego indican que el jugador debe robarla.

Lo que dice el paso recursivo de esta función es que el jugador debe robar cartas de forma tal que minimize los puntos que pueda ganar el contrincante, sabiendo que este también jugará de forma óptima. Por ejemplo, si el valor mínimo de la función $opt$ es $opt(i+3,j)$ significa que el jugador debe robar las primeras 3 cartas empezando por la izuquierda.

\subsection{Implementación:}

Para implementar un algoritmo que utiliza la técnica de programación dinámica necesitamos de una estructura para guardar los resultados que vamos computando para no tener que recalcularlos, y de esta forma lograr una buena complejidad temporal.

Para resolver este algoritmo utilizamos una matriz de $n * n$ donde $n$ es la cantidad de cartas al inciar el juego. Durante este informe llamaremos a la matriz $msp$.

El valor de la casilla $msp[i][j]$ será por un lado la cantidad de puntos óptima que se podrá lograr a partir de las cartas que quedan ($c_i ... c_j$) y por otro lado dos valores que indican las cartas que quedarán luego de que el jugador que le toque realize su turno, estos valores se utilizan para que dada una secuencia de cartas se pueda realizar un seguimiento de todos los turnos que se realizen hasta terminar el juego.

Por ejemplo si en la casilla $msp[i][j]$ tiene los valores $i+k, j$ significa que en ese turno el jugador robo k cartas empezando desde la izquierda y la proxima casilla que se debe obserbar para continuar con el seguimiento del juego es la casilla $msp[i+k][j]$. Además, si los valores de la casilla son $-1, -1$ significa que en ese turno el jugador robó todas las cartas restantes, terminando de esta forma el juego.



\subsection{Complejidad temporal}
\label{problema3-complejidad}
Nosotros afirmamos que nuestro algoritmo tiene una complejidad de $O(n^3 * k)$ operaciones. Vamos a demostrar esto de a partes.

Antes de empezar, vamos a hacer unas definiciones de tipo, válidas para los pseudocódigos de esta sección.
\begin{pseudo}
\State Tipo de dato Casillero es Tupla $\langle$ fila : entero, columna : entero, k : entero $\rangle$
\State Tipo de dato Entrada es Tupla $\langle$ n : entero, k\_max : entero, tablero : vector $\langle$ vector $\langle$ entero $\rangle$ $\rangle$, origen : Casillero, destino : Casillero $\rangle$
\State Tipo de dato Salida es Tupla $\langle$ saltos : entero, caminoMinimo : list $\langle$ Casillero$\rangle$ $\rangle$
\end{pseudo}

Desde un punto de vista mas ``macro'', nuestro algoritmo hace lo siguiente:

\begin{pseudo}
    \Procedure{Saltos en la \emph{Matrix}}{}
        \State Entrada E $\leftarrow$ leerEntrada() \Ode{n^2}
        \State Salida S(V) $\leftarrow$ resolver(E) \Ode{n^3*k}
        \State Imprimir(S) \Ode{n}
        \State \textbf{return} 0
    \EndProcedure
\end{pseudo}

\textbf{leerEntrada()} se encarga de cargar los datos vía \emph{stdin}. Por esto, realiza una cantidad constante de asignaciones todas las cuales son $O(1)$ ya que los tipos de datos son enteros o Casilleros (los cuales se componen de tres enteros). Una vez cargados los datos, creamos el tablero. El tablero es vector $\langle$ vector $\langle$ entero $\rangle$ $\rangle$ que representa el campo de juego. Por esto, tiene tamaño $n*n$ y realiza justamente $n^2$ asignaciones de enteros. Sumando todo esto, \textbf{leerEntrada()} tiene un costo de $O(n^2)$.

Sin embargo, cabe aclarar que después realizamos una asignación (ya que la Entrada E no es pasada por referencia). Esta asignación de Entrada se compone de tres asignaciones de enteros ($O(1)$ cada una), un asignación de tablero ($O(n^2)$) y dos asignaciones de Casillero ($O(1)$ cada uno). Pero esto no cambia el costo total de $O(n^2)$.

\textbf{resolver(E)} es el núcleo de nuestro algoritmo y es lo que más cuesta, y por esta razón lo vamos analizar con más detalle.

\begin{pseudo}
\State Tipo de dato Matriz3D es vector $\langle$ vector $\langle$ vector $\langle$ Tupla $\langle$ Casillero, k $\rangle$ $\rangle$ $\rangle$ $\rangle$
    \Procedure{resolver}{$Entrada E$}
        \State Matriz3D yaPase $\leftarrow$ crearMatriz() \Ode{1}
		\Comment{Empieza BFS, pero modificado para nuestra conveniencia}
		\State q $\leftarrow$ crearCola() \Ode{1}
		\Comment{La funcion que sigue marca al casillero e.origen en yaPase con cantidad de saltos 0}
		\State marcoComo(e.origen, yaPase, 0) \Ode{1}
		
		\State encolar(e.origen, q) \Ode{1}
		
		\While{!q.empty()} \Ode{1}
			\State c $\leftarrow$ q.desencolar() \Ode{1}
			\If{c == destino}
				\State break
			\EndIf
			
			\For{cada Casillero u en losAdyacentes(e,c)} \Ode{n}
			    \If{!estaEn(u,yaPase)} \Ode{1}
					\State marcoComo(u, yaPase, cantSaltos(c) + 1) \Ode{1}
					\State encolar(u, q) \Ode{1}
				\EndIf
      		\EndFor
		\EndWhile

      	\Comment{Terminamos BFS y pasamos a escribir la salida. escriboSalida en nuestro codigo no es una funcion aparte, pero para este pseudocodigo lo decidimos dejar así.}
      	\State escriboSalida(S, yaPase) \Ode{n}
      	\State \textbf{return} S
        
    \EndProcedure
\end{pseudo}

La parte importante de resolver, es el BFS que realizamos, ya que las otras partes del algoritmo tienen complejidad menor.

Vamos a empezar diciendo que un casillero cualquiera puede tener a lo sumo $2*(n-1)$ casilleros adyacentes (toda su fila y toda su columna, menos él mismo). Por esto, cuando realizamos losAdyacentes, buscamos a lo sumo $O(n)$ adyacentes. Si ya sabemos que estamos fuera del rango del tablero, no vale la pena seguir buscando. Por esta misma razón, el ciclo for que utiliza los adyacentes corre a lo sumo $O(n)$ veces por cada casillero.

Teniendo en cuenta esto, vamos a correr el ciclo while grande a lo sumo una vez por casillero. Cabe recordar que tenemos $n*n*k$ casilleros en total: un casillero por cada celda del tablero, y ese tablero lo tenemos k veces. Lo peor que puede suceder es tener que correr el while para cada casillero, por lo que tendríamos un costo total de $O(n*n*n*k)$, que es igual a $O(n^3*k)$.

Al escribir la salida, escribimos la cantidad de saltos, y cuáles fueron dichos saltos, osea cuál fue el camino que se tomó desde el inicio hasta el dstino. Podemos observar que la cantidad de saltos es $O(n)$. En el peor caso, todos los casilleros tienen potencia 1 (lo menor que pueden valer) y el jugador no tiene potencias extras para gastar ($k=0$). En este caso, estamos en el primer ejemplo de la sección  \ref{problema3-descripcion}. En este caso, una solución óptima es recorrer desde la fila origen hasta la fila destino, y luego de la columna origen hasta la columna destino. Tanto las filas como las columnas pueden estar separadas a lo sumo por $n-1$ casilleros. Por esto, el camino más corto esta claramente acotado por $2*n$, por lo que tenemos a lo sumo $O(2*n)$ iteraciones de ciclo. Por todo esto, podemos afirmar que \textbf{Imprimir(S)} tiene una complejidad de $O(n)$

Finalmente, \textbf{Imprimir(S)} es la función encargada de imprimir el resultado por pantalla. Vamos a utilizar un pseudocodigo para probar su complejidad.

\begin{pseudo}
    \Procedure{Imprimir}{$Salida s$}
        \State imprimir(s.saltos) \Ode{1}

		\For{cada Casillero c en s.caminoMinimo} \Ode{1}
        \State imprimir(c.fila) \Ode{1}
        \State imprimir(c.columna) \Ode{1}
        \State imprimir(kUsado(c, padre(c))) \Ode{1}
        \Comment{Este kUsado en nuestro algoritmo en C++ lo obtenemos sabiendo cual era nuestro padre con un iterador auxiliar. Como esto es pseudocodigo decidimos dejarlo de esta manera}
      	\EndFor
        
    \EndProcedure
\end{pseudo}

Como se puede observar, todas las impresiones por pantalla son $O(1)$ y la guarda del for también cuesta $O(1)$. Sin embargo, falta justificar cuantas veces corre el ciclo for. Como probamos anteriormente, el camino está acotado por $2*n$. Por esto, realizamos $O(n)$ cantidad de ciclos for.

Sumando todas las complejidades, llegamos a obtener una complejidad temporal total de $O(n^3 * k)$, con lo que cumplimos con la complejidad temporal pedida.

\subsection{Demostración de correctitud}
\label{problema3-demostracion}
Como dijimos anteriormente lo que se busca en encontrar un camino de mínimas aristas entre el casillero de entrada con $k_{max}$ osea el estado $(x_e,y_e,k_{max})$ hasta la salida sin importar las unidades de potencia restante, osea $(x_s,y_s,i)$ para algún i.

Como sabemos si realizamos BFS desde un nodo v en el grafo G, para cada nodo u $\in$ G, encuentra el camino de mínimas aristas entre v y u. Por eso al aplicar BFS desde $(x_e,y_e,k_{max})$ encontramos el camino de mínimas aristas entre él y la salida. Lo único que queda hacer es elegir entre los caminos mínimos de los estados  $(x_s,y_s,i)$ cual es el que tiene menos aristas y esa es la solución. 

\subsection{Experimentación}
\label{problema3-experimentacion}
\emph{Nota:} Como vamos a hablar mucho de las distintas podas vamos a denominarlas de la siguiente manera:
\begin{itemize}
\item Poda 1: Si aunque pongamos piezas en los casilleros que me quedan revisar, no llegamos a alcanzar la cantidad de piezas de la solución óptima encontrada a ese momento, corto.
\item Poda 2: Similar a la poda 1, pero además revisamos hasta $m$ casillas para adelante teniendo en cuenta las restricciones ya existentes en el tablero.
\end{itemize}

En este ejercicio, vamos a enfocar la experimentación orientándolo a la comparación entre las distintas podas, así como también comparar las mismas contra no usar ninguna poda. Cabe aclarar que no usar ninguna poda tiene un método de corte \emph{naïf}: cortar si logré llenar el tablero.

Primero probamos todos los casos posibles para n = 1 a n = 3 sin poda y para las dos podas que propusimos.

En los siguientes gráficos podemos observar que utilizar podas mejora drásticamente el tiempo de ejecución, respecto a no utilizar podas. No logramos realizar todos los casos de 3*3 sin podas ya que necesitaba un tiempo muy grande.

\begin{center}
  \begin{figure}[H]
    \includegraphics[width=0.5\linewidth]{problema3/graficos/comparacion_podas_todas_instancias_llamados.eps}
    \includegraphics[width=0.5\linewidth]{problema3/graficos/comparacion_podas_todas_instancias_tiempo.eps}
    \caption{Comparación con y sin podas. Izquierda: llamados recursivos, Derecha: tiempo}
    \label{fig:problema3-sin-y-con-podas}
  \end{figure}
\end{center}

En estos gráficos podemos observar que utilizar podas mejora tanto en la cantidad de llamados recursivos, así como también mejora en tiempo de ejecución. A su vez, si bien la cantidad de llamados recursivos que realiza el algoritmo al utilizar la poda 2 es menor a la cantidad de llamados recursivos que realiza al utilizar la poda 1, como el n es chico la cantidad de casos que poda no logra contrarestar el overhead que genera por lo que el tiempo de ejecución del algoritmo utilizando poda 2 es claramente mayor a utilizar la poda 1.

No usar podas es tan malo porque revisa todos los tableros posibles, y recién una vez revisados todos termina (salvando el caso en que justo encuentre el tablero completo). Esto implica que ya para tableros de 3*3 es un número grande de tableros posibles, y para tableros aún más grandes el algoritmo tarda una cantidad incomparable de tiempo con respecto a las podas. Por esta razón, sin podas no esta presente en el tercer gráfico de arriba, ni en los gráficos de abajo.

A continuacion experimentamos con 50 casos de 4*4 generados pseudo-aleatoriamente pero con una semilla constante. Cabe aclarar que se comparan los resultados de las dos podas para la misma entrada en cada caso.

\begin{center}
  \begin{figure}[H]
    \includegraphics[width=\linewidth]{problema3/graficos/comparacion_podas1y2_llamados_recursivos.eps}
    \caption{Comparación de podas, según llamados recursivos}
    \label{fig:problema3-podas-llamados}
  \end{figure}
\end{center}

\begin{center}
  \begin{figure}[H]
\includegraphics[width=\linewidth]{problema3/graficos/comparacion_podas1y2_tiempo.eps}
    \caption{Comparación de podas, según tiempo de ejecución}
    \label{fig:problema3-podas-tiempo}
  \end{figure}
\end{center}


En el gráfico \ref{fig:problema3-podas-llamados}, podemos observar claramente que la poda 2 logra podar mas casos y por consiguiente, realiza menos llamados recursivos. Vemos que al aumentar el n, la cantidad de casos que logra podar aumenta, logrando en la mayoria de los casos superar el overhead como se ve en la figura \ref{fig:problema3-podas-tiempo}. Sin embargo cabe aclarar que no siempre logra compensar lo suficiente el costo adicional de $O(m)$.

Como ejemplo, el caso 35 logra podar lo suficiente para lograr 4 veces menos llamados recursivos. Sin embargo, logra solamente ser un 25\% (aproximadamente) mejor en tiempo de ejecución.

Por otro lado, la poda 1 también tiene beneficios. Es considerablemente más fácil de implementar que la poda 2. A su vez, la poda 1 no depende de la cantidad de colores existentes, mientras que la poda 2 sí lo hace. La poda 2 crea un $vector<int>$ donde la cantidad de posiciones se corresponde con la cantidad de colores existentes. Si bien no es un problema muy grave, puede ocurrir que sí lo sea si tenemos una computadora con memoria acotada y tenemos un caso donde la cantidad de colores es muy alta, como por ejemplo un millón de colores.

Curiosamente, si bien utilizar la poda 1 incurre en una menor cantidad de tiempo de ejecución para casos mas chicos (como por ejemplo tableros de tamaño de 3*3), esto no se mantiene para casos más grandes (como tableros de tamaño 4*4). Es decir, la poda 2 comienza a tener más importancia cuando el tablero es más grande. Creemos que esto se debe a que las podas que realiza la poda 2 tienen más impacto con respecto al costo de realizar dicha poda en tableros de tamaño más grande.

\subsection{Conclusión}
\label{problema3-resultados}
Para concluir, nos parece importante destacar que la complejidad temporal de nuestra solución es dominada por la etapa de ordenamiento, lo cual provee una cota inferior a la complejidad temporal de nuestra solución, debido a que los algoritmos de ordenamiento por comparaciones son $\Omega(n * log(n))$. Sin embargo, en el caso en que los datos de entrada se encuentren ordenados o dentro de un rango acotado\footnote{Considerar casos donde la entrada solo admite días dentro de un rango es razonable dado el dominio del problema.}, la instancia se puede resolver en tiempo $O(n)$.

%  esto lo comento porque la etapa de lectura de datos no la incluímos en el análisis teórico, y no estoy seguro de que es imposible resolver el problema en tiempo mejor que O(n) para entradas ordenadas.
% Por otro lado, sin contar la etapa de ordenamiento, realizamos a lo sumo $O(n)$ operaciones, ya que podríamos tener que recorrer el vector por lo menos dos veces: una para guardar los datos y otra para saber cuál es el día óptimo para colocar al inspector. Por esto, podemos asegurar que por más que los datos vengan ordenados no vamos a poder mejorar la complejidad temporal de $O(n)$.

\newpage


%%%%%%%%%%%%%%%%%%%%%%%%%%%%%%%%%%%%%%%%%%%%%%%%%%%%%%%%%%%%%%%%%%%%%%%%%%%%%%%
%% Apéndices                                                                 %%
%%%%%%%%%%%%%%%%%%%%%%%%%%%%%%%%%%%%%%%%%%%%%%%%%%%%%%%%%%%%%%%%%%%%%%%%%%%%%%%

\begin{appendices}

\section{Código fuente del problema 1}
\label{problema1-codigo}

\verbatiminput{./codigo/problema1.cpp}

\newpage


\section{Código fuente del problema 2}
\label{problema2-codigo}
\verbatiminput{./codigo/problema2.cpp}

\newpage


\section{Código fuente del problema 3}
\label{problema3-codigo}
\verbatiminput{./codigo/problema3.cpp}


\end{appendices}

\end{document}