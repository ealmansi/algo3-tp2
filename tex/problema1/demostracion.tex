Para demostrar que nuestra solución resuelve el problema propuesto tenemos que demostrar dos cosas. La primera es que la función recursiva que nosotros propusimos realmente determina los movimientos óptimos para realizar y la segunda es que la matriz que usamos para guardar las subsoluciones es completada correctamente.

Primero vamos a demostrar que la función recursiva determina las jugadas óptimas del juego. 

Para empezar recordemos lo que lo que el problema pide es jugar de una forma que maximice tu puntuación. Como al finalizar el juego no pueden quedar cartas sin robar, la suma de los valores de todas las cartas va a estar distribuida entre el jugador uno y el jugador dos, de esta forma:

$\sum_{i=0}^{n} c_i = \sum_{j \in A} c_j + \sum_{k \in B} c_k$

Donde A es el conjunto de cartas que robó el jugador uno y B es el conjunto de cartas que robó el jugador dos.

Por lo tanto mientras más chico sea el valor de la suma de los valores del conjunto B, el valor del conjunto A va a crecer, y viceversa, ya que se da la siguiente igualdad:

$\sum_{i=0}^{n} c_i  - \sum_{j \in B} c_j= \sum_{k \in A} c_k$

Supongamos que somos el jugador uno. Lo que el problema pide es que juguemos de forma óptima, por lo tanto debemos maximizar nuestra puntuación. Maximizar nuestra puntuación es maximizar el valor de la suma de los elementos de A, y  por lo que acabamos de ver esto es lo mismo que minimizar la suma de los elementos de B, que representa la puntuación del jugador dos.

Recordemos que nuestra función recursiva es la siguiente:

$opt(i,i) = c_i$ \\
$opt(i,j) = \sum cartas - min(opt(i+1, j), ..., opt(j,j), opt(i, j-1), ... ,opt(i,i), 0)$

Por lo dicho anteriormente demostrar que nuestra función resuelve el problema es lo mismo que demostrar que nuestra función minimiza el puntaje del jugador contrario.

Vamos a demostrar esto por inducción en las cartas restantes, que en este caso son $j-i+1$. La propiedad que queremos demostrar es la siguiente:

$P(i,j) : (\forall i,j, 0 \leq i \leq j < n)opt(i,j)$ maximiza nuestro puntaje para las cartas desde $c_i$ hasta $c_j$

Primero analicemos el caso base:

$opt(i,i) = c_i$

Queremos ver que se cumple $P(i,i)$.

Este caso significa que sólo queda una carta. Como el juego nos obliga a, de ser posible, robar al menos una carta por turno, la única opción posible es robar esta carta. Como es la última opción, también es la opción óptima, por lo tanto se cumple $P(i,i)$.

Ahora analicemos el caso recursivo:

$opt(i,j) = \sum cartas - min(opt(i+1, j), ..., opt(j,j), opt(i, j-1), ... ,opt(i,i), 0)$

Tenemos que realizar el paso inductivo, esto significa que tenemos que probar que $P(i,j)$ seguro vale si vale $P(i',j')$ para todas las subsecuencias de cartas que $opt(i,j)$ utiliza, osea, para probar $P(i,j)$ queremos ver que:

$[(\forall i',j',(( i < i' \wedge j = j') \vee (i' = i \wedge j' < j))) P(i',j')]  \implies P(i,j)$

\textbf{HI:} $((\forall i',j',(( i < i' \wedge j = j') \vee (i' = i \wedge j' < j))) P(i',j'))$

Para demostrar esto vamos a dar como válida la hipótesis inductiva y a partir de eso probar que vale $P(i,j)$.

En este caso quedan $j-i+1$ cartas, tenemos que demostrar que $opt(i,j)$ toma la decisión que minimiza el puntaje del jugador contrario. Como el problema nos dice que tenemos que asumir que el jugador contrario también juega de forma óptima, entonces su puntaje también estará determinado por la función $opt$.

Lo que la función debe determinar es, de las posibles instancias del juego que se pueden generar luego de jugar nuestro turno, la que minimiza la función de $opt$ ya que el puntaje del jugador contrario será el resultado de aplicar $opt$ a la instancia que quede. Y por hipótesis inductiva sabemos que al aplicar $opt$ a dicha instancia se obtendrá un resultado óptimo.

Como sólo se pueden robar cartas de uno de los dos extremos las posibles instancias son las siguientes: Si se roba una carta comenzando de la izquierda la instancia generada será de $(i+1,j)$, si se roban 2, de $(i+2,j)$ y se se roban $j-i$, de $(j,j)$. De esta misma forma si se roba una carta empezando de la derecha quedará una instancia de $(i,j-1)$ y si se roban $j-i$ cartas, la instancia sera de $(i,i)$.

Pero queda una instancia más por analizar, la de levantar todas las cartas posibles (no importa de cual extremo). En ese caso se generará una instancia $(0,0)$.

Por lo tanto el mínimo que debemos hallar es el siguiente:

$min(opt(i+1,j), ... , opt(j,j), opt(i,j-1), ... , opt(i,i), 0)$

Y la puntuación óptima se determina por la cuenta:

$\sum cartas - min(opt(i+1,j), ... , opt(j,j), opt(i,j-1), ... , opt(i,i), 0)$

Por lo tanto, si asumimos que se cumple nuestra hipótesis inductiva, nuestra función $opt$ minimiza el puntaje del jugador contrario. Osea que realmente determina nuestra puntuación óptima y el movimiento a realizar para obtenerla, que es lo que el problema pide.

Lo que queda mostrar es que completamos la matriz de forma correcta. Como cada casilla de la matriz es un subproblema y para resolver cada subproblema usamos subproblemas mas chicos, tenemos que mostrar que a la hora de completar una casilla siempre tenemos completas las casillas de los subproblemas que se necesitan.

La matriz, tal y como dijimos en el desarrollo, se completa de arriba a abajo y de izquierda a derecha.

Entonces, supongamos que queremos llenar la casilla $opt[i][j]$. Como la matriz se llena de abajo hacia arriba, a la hora de llenar esta casilla ya van a estar llenas las casillas $opt[i][j']$ donde $j < j'$. Y como la matriz se llena de izquierda a derecha, a la hora de completar esta casilla ya van a estar completas las casillas $opt[i'][j]$, donde $i' < i$.

Además, como dijimos en el desarrollo para llenar una casilla $opt[i][j]$ se necesita observar los valores de las casillas que están a la izquierda de la ésta y las que están abajo. Como se muestra en la Figura 1.

Osea que las casillas necesarias son $opt[i'][j]$, donde $i' < i$ y $opt[i][j']$ donde $j < j'$. Estas son las mismas casillas que ya dijimos que iban a estar completas a la hora de completar $opt[i][j]$. Por lo tanto al completar $opt[i][j]$ casilla van a estar previamente completadas las casillas necesarias para hacerlo, por lo tanto nuestra matriz se llena correctamente.