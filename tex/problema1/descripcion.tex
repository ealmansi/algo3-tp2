Este problema trata de un juego de cartas llamado \emph{Robanúmeros}. El mismo es un juego para dos jugadores. Al empezar el juego se cuenta con una secuencia de cartas, todas ellas boca arriba, cada carta tiene un valor numérico entero (puede ser positivo o negativo). Este juego se juega por turnos de forma alterna y en cada turno un jugador debe elegir un extremo de la secuencia de cartas y robar una cantidad de las mismas empezando desde ese extremo. Por ejemplo supongamos que la secuencia de cartas es la dada en la siguiente tabla:
\begin{center}
\begin{tabular}{|c|c|c|c|c|}
\hline
2 & -3 & 7 & 8 & -10 \\
\hline
\end{tabular}
\end{center}

Dada esta secuencia, el primer jugador en jugar puede, por ejemplo, elegir el extremo derecho y robar 3 cartas, quedándose así con -10, 8 y 7. También podría elegir 4 cartas por el lado izquierdo y de esta forma se quedaría con 2, -3, 7 y 8. Sin embargo, el jugador no puede robar 2, -3 y 8, o -10, y -3.

En cada turno el jugador debe robar al menos una carta y al no quedar más, el jugador que obtenga el número más alto al sumar los valores de todas las cartas que robó es el ganador.

En este problema debemos diseñar un algoritmo que juegue a este juego de manera óptima. Este algoritmo debe estar pensado para jugar contra otro jugador que también juegue de forma óptima. La definición de que un jugador juegue de manera óptima significa que la diferencia de puntos obtenida a su favor sea la mayor diferencia que se puede obtener frente a un oponente que también juega de la misma forma ante cada situación que se le deje.

El algoritmo debe tener una complejidad temporal de peor caso de $O(n^3)$ donde $n$ es la cantidad de cartas en la secuencia inicial.

\subsubsection{Ejemplos y observaciones}

A continuación vamos a dar algunos ejemplos con sus soluciones y algunos casos particulares que pensamos.

Un ejemplo se da ante la siguiente secuencia de cartas:

\begin{center}
\begin{tabular}{|c|c|c|c|c|}
\hline
2 & 3 & 7 & 8 & -10 \\
\hline
\end{tabular}
\end{center}

La forma óptima de jugar para el jugador que empieza es robar todas las cartas empezando desde la izquierda excepto la última, luego de ese turno el segundo jugador deberá robar la carta con el valor $-10$. Al final del juego el jugador uno tendrá un puntaje de $20$ y el jugador dos, de $-10$.

Otro ejemplo es si todas las cartas son positivas. En ese caso el jugador uno debe robar todas las cartas para obtener la mayor diferencia posible.

Un caso interesante el siguiente, sería tener en el medio una carta de valor muy bajo (menor a cero) y a ambos lados cartas de valor parecido. Podría ser:

\begin{center}
\begin{tabular}{|c|c|c|c|c|c|c|}
\hline
1 & 1 & -100 & 1 & 1 \\
\hline
\end{tabular}
\end{center}

En este caso el que robe la carta del medio va a perder ya que tiene un valor muy bajo. Y si ambos juegan óptimo el que deba robar esa carta será el jugador 1. Ya que ambos jugadores irán robando de a una carta y el jugador uno va a terminar teinendo que robar la carta con valor negativo.