Para resolver este problema diseñamos un algoritmo que utilize la técnica de programación dinámica. Lo primero que hicimos fue pensar la función recursiva que utilizará el algoritmo para detérminar cual es el movimiento óptimo para hacer. La signatura de la función será:

$opt(i,j) = c$

Donde $C$ es el total de los puntos que sumamos e $i$ y $j$ son los indices de las cartas que quedan en juego, $i$ es la primer carta que queda desde la izquierda, y $j$ es la primer carta desde la derecha.

\subsection{Función Recursiva:}

Pensamos la función según el siguiente enfoque: Al finalizar el juego, la suma de las cartas que robamos será la suma de todas las cartas menos la suma de las cartas que robo el rival, esto es así porque según las reglas del juego no podemos elegir no robar cartas:

$\sum{i=0}^{n} c_i = \sum{j \in A} c_j + \sum{k \in B} c_k$

Donde A es el conjunto de cartas que robó el jugador A y B es el conjunto de caartas que robó el jugador B.

Lo que pensamos es que maximizar la sumatoria de puntos que robemos nosotros es lo mismo que minimizar la suma de puntos robada por el otro jugador. Además sabemos que el otro jugador también jugara de forma óptima, por lo tanto llegamos a la conclusión de que la función recusrsiva para resolver este problema es la siguiente:

$opt(0,0) = 0$
$oppt(i,i) = c_i$
$opt(i,j) = \sum cartas - min(opt(i+1, j), ..., opt(j,j), opt(i, j-1), ... ,opt(i,i)$

Vale aclarar que por cómo pensamos la función, siempre pasa que $i \leq j$.

Los casos base ocurren cuando no queda ninguna carta, o cuando queda una sola carta. Al no quedar cartas el jugador no puede ganar ningún punto, y al quedar una sola carta, las reglas del juego indican que el jugador debe robarla.

Lo que dice el paso recursivo de esta función es que el jugador debe robar cartas de forma tal que minimize los puntos que pueda ganar el contrincante, sabiendo que este también jugará de forma óptima. Por ejemplo, si el valor mínimo de la función $opt$ es $opt(i+3,j)$ significa que el jugador debe robar las primeras 3 cartas empezando por la izuquierda.

\subsection{Implementación:}

Para implementar un algoritmo que utiliza la técnica de programación dinámica necesitamos de una estructura para guardar los resultados que vamos computando para no tener que recalcularlos, y de esta forma lograr una buena complejidad temporal.

Para resolver este algoritmo utilizamos una matriz de $n * n$ donde $n$ es la cantidad de cartas al inciar el juego. Durante este informe llamaremos a la matriz $msp$.

El valor de la casilla $msp[i][j]$ será por un lado la cantidad de puntos óptima que se podrá lograr a partir de las cartas que quedan ($c_i ... c_j$) y por otro lado dos valores que indican las cartas que quedarán luego de que el jugador que le toque realize su turno, estos valores se utilizan para que dada una secuencia de cartas se pueda realizar un seguimiento de todos los turnos que se realizen hasta terminar el juego.

Por ejemplo si en la casilla $msp[i][j]$ tiene los valores $i+k, j$ significa que en ese turno el jugador robo k cartas empezando desde la izquierda y la proxima casilla que se debe obserbar para continuar con el seguimiento del juego es la casilla $msp[i+k][j]$. Además, si los valores de la casilla son $-1, -1$ significa que en ese turno el jugador robó todas las cartas restantes, terminando de esta forma el juego.

