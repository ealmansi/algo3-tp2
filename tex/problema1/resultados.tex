Para concluir nos parece importante que ante un problema de optimización, que en este caso fue abordado utilizando la técnica de programación dinámica, siempre es necesario contemplar todos los posibles movimientos y todos los posibles subproblemas que estos movimientos puedan generar para poder hallar la solución óptima. También nos parece importante que el beneficio de utilizar la técnica de programación dinámica es el hecho de contar con un diccionario (en este caso una matriz) que tenga ya resueltos los subproblemas, evitando calcular las soluciones de los mismos más de una vez.