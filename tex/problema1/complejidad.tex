En la Figura 2, que muestra el pseudocódigo de nuestro algoritmo situada arriba podemos ver que la complejidad de este algoritmo se representa con esta cuenta:

$O(n) + 2*O(1) + n*(2*O(1) + j*(2*O(1) +O(n) + 6*O(1)) + O(n)$

Como $0 \leq j \leq n$ entonces $j$ está acotado superiormente por $n$. Entonces:

$O(n) + 2*O(1) + n*(2*O(1) + n*(2*O(1) +O(n) + 6*O(1)) + O(n)$

Por álgebra de órdenes esto se puede reducir a:

$O(2n+2) + n*(O(2) + n*(O(2) + O(n) + O (6)))$

$O(n) + n*(O(1) + n*(O(n+1)))$

$O(n) + n*(O(1) + O(n^2+n))$

$O(n) + n*(O(n^2 +n +1))$

$O(n) + n*O(n^2)$

$O(n^3+n)$

$O(n^3)$

Entonces podemos decir que la complejidad de nuestro algoritmo es $O(n^3)$, pero primero tenemos que demostrar que los algoritmos $escribirSalida$ y $buscarMinimo$ realmente tienen una complejidad lineal.

El algoritmo de $buscarMinimo$ es el siguiente:

\begin{center}
 \begin{figure}[H]
  \begin{pseudo}
   \Procedure{buscarMinimo}{opt,i,j,iSiguiente,jSiguiente,minCantPuntos}
    \State $k \leftarrow 1$\Ode{1}
    \While{$k \leq j-i$} \hfill $(j-i)*O(1)$
      \If{$opt[i+k][j].cantPuntos < minCantPuntos$ }\Ode{1}
	\State $iSiguiente \leftarrow i+k$\Ode{1}
	\State $jSiguiente \leftarrow j$\Ode{1}
	\State $minCantPuntos \leftarrow opt[i+k][j].cantPuntos$\Ode{1}
      \EndIf
      \If{$opt[i][j-k].cantPuntos < mincantPuntos$}\Ode{1}
	\State $iSiguiente \leftarrow i$\Ode{1}
	\State $jSiguiente \leftarrow j-k$\Ode{1}
	\State $mincantPuntos \leftarrow opt[i][j-k].cantPuntos$\Ode{1}
      \EndIf
    \EndWhile
   \EndProcedure
  \end{pseudo}

 \end{figure}

\end{center}

La complejidad del algoritmo es:

$O(1) + (j-i)*(8*O(1))$

Sabemos que $0 \leq i \leq j-1$ y $0 \leq j \leq n-1$, entonces $0 \leq j-i \leq n-j$. Y como j esta acotado inferiormente por cero, entonces $n-j \leq n$. Finalmente:

$0 \leq j-i \leq n$

Entonces la complejidad del algoritmo está acotada por:

$O(1) + n*(8*O(1))$

Por álgebra de órdenes:

$O(1) + O(8*n)$

$O(n)$

Ahora veamos que el algoritmo $escribirSalida$ también tiene complejidad lineal.

Vale aclarar que en este algoritmo usamos una estructura llamada $Salida$, que tiene un campo llamado $cantTurnos$ que indica la cantidad total de turnos hasta que se acaba el juego, un campo llamado $turnos$ que es un vector que indica todos los turnos que hubo, y dos campos que indican la cantidad de puntos que hizo cada jugador, sus nombres son $puntajeJ1$ y $puntajeJ2$.

El pseudo código del algoritmo es el siguiente:

\begin{center}
 \begin{figure}[H]
  \begin{pseudo}
   \Procedure{escribirSalida}{opt, sumasParciales, n}
    \State $Salida$  s \Ode{1}
    \State $i \leftarrow 0, j \leftarrow n-1$ \Ode{1}
    \State $turno \leftarrow opt[i][j]$ \Ode{1}
    \While{$turno.iSiguiente \neq -1 \wedge turno.jSiguiente \neq -1$} \hfill $n*O(1)$
      \If{$0 < turno.iSuguiente -i$} \Ode{1}
	\State $s.turnos.agregar(\langle izq, turno.iSiguiente - i \rangle)$ \Ode{1}
      \Else 
	\State $s.turnos.agregar(\langle der, j - turno.jSiguiente \rangle)$ \Ode{1}
      \EndIf
      \State $i \leftarrow turno.iSiguiente, j \leftarrow turno.jSiguiente$ \Ode{1}
      \State $turno \leftarrow opt[i][j]$ \Ode{1}
    \EndWhile
    \State $s.turnos.agregar(\langle izq, j - turno.jSiguiente \rangle)$ \Ode{1}
    \State $s.cantTurnos \leftarrow s.turnos.size()$ \Ode{1}
    \If{$s.cantTurnos$ mod $2 = 0$} \Ode{1}
      \State $s.puntajeJ1 = opt[0][n-1].cantPuntos$ \Ode{1}
      \State $s.puntajeJ2 = sumasParciales[n-1] - s.puntajeJ1$ \Ode{1}
    \Else
      \State $s.puntajeJ2 = opt[0][n-1].cantPuntos$ \Ode{1}
      \State $s.puntajeJ1 = sumasParciales[n-1] - s.puntajeJ2$ \Ode{1}
    \EndIf
    \State $\textbf{return}$ s \Ode{1}
   \EndProcedure
  \end{pseudo}

 \end{figure}

\end{center}

Vale aclarar que al haber $n$ cartas, la máxima cantidad de turnos posible es $n$. El juego tendrá $n$ turnos cuando cada jugador robe de a una carta por turno.

La complejidad de este algoritmo es:

$3*O(1) + n*(5*O(1)) + 8*O(1)$

$O(1) + O(n) + O(1)$

$O(n+2)$

$O(n)$
