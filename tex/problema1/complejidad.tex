En la figura que muestra el pseudocódigo de nuestro algoritmo situada arriba podemos ver que la complejidad de este algoritmo se representa con esta cuenta:

$O(n) + 2*O(1) + n*(2*O(1) + n*(2*O(1) +O(n) + 6*O(1)) + O(n)$

Por álgebra de órdenes esto se puede reducir a:

$O(2n+2) + n*(O(2) + n*(O(2) + O(n) + O (6)))$

$O(n) + n*(O(1) + n*(O(n+1)))$

$O(n) + n*(O(1) + O(n^2+n))$

$O(n) + n*(O(n^2 +n +1))$

$O(n) + n*O(n^2)$

$O(n^3+n)$

$O(n^3)$

Entonces podemos decir que la complejidad de nuestro algoritmo es $O(n^3)$, pero primero tenemos que demostrar que los algoritmos $escribirSalida$ y $buscarMinimo$ realmente tienen una complejidad lineal.

El algoritmo de $buscarMinimo$ es el siguiente

\begin{center}
 \begin{figure}[H]
  \begin{pseudo}
   \Procedure{buscarMinimo}{opt,i,j,iSiguiente,jSiguiente,minCantPuntos}
    \State $k \leftarrow 1$\Ode{1}
    \While{$k \leq j-1$} \hfill $n*O(1)$
      \If{$opt[i+k][j].cantPuntos < minCantPuntos$ }\Ode{1}
	\State $iSiguiente \leftarrow i+k$\Ode{1}
	\State $jSiguiente \leftarrow j$\Ode{1}
	\State $minCantPuntos \leftarrow opt[i+k][j].cantPuntos$\Ode{1}
      \EndIf
      \If{$opt[i][j-k].cantPuntos < mincantPuntos$}\Ode{1}
	\State $iSiguiente \leftarrow i$\Ode{1}
	\State $jSiguiente \leftarrow j-k$\Ode{1}
	\State $mincantPuntos \leftarrow opt[i][j-k].cantPuntos$\Ode{1}
      \EndIf
    \EndWhile
   \EndProcedure
  \end{pseudo}

 \end{figure}

\end{center}

La complejidad del algoritmo es:

$O(1) + n*(8*O(1))$

Por álgebra de órdenes:

$O(1) + O(8*n)$

$O(n)$

Ahora veamos que el algoritmo $escribirSalida$ también tiene complejidad lineal:

%Continuará
