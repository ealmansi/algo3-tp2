Si bien el problema tratado pareciera a primera vista admitir una solución en la forma de programación dinámica, el hecho de tener un recurso agotable en consideración no permite aplicar el principio de optimalidad. Es decir, es posible que en un camino óptimo se llegue a un casillero intermedio de forma subóptima (posiblemente para ahorrar unidades de salto extra).

De todas formas, el problema puede ser resuelto eficientemente una vez que se modela apropiadamente mediante grafos, utilizando el algorítmo de búsqueda BFS. En este caso, la representación de cada instancia no es tan trivial como asignar simplemente un nodo a cada casillero de la matriz, sino fue necesario desdoblar cada uno de ellos en $k$ nodos distintos. De esta forma, incorporamos dentro del grafo la información referente a las unidades de salto extra. Esto pone en evidencia que al modelar un problema, la representación de una instancia no necesariamente tiene que ser un mapeo directo o natural de los componentes del mismo.

Por otro lado, al haber dos variables $n$ y $k$ influyendo en el costo de la resolución del problema, la experimentación realizada resulta insuficiente para corroborar el resultado de complejidad temporal obtenido teóricamente. Logramos verificar empíricamente dentro del rango analizado que la función de costo muestra un comportamiento cúbico respecto a $n$ y lineal respecto a $k$. Sin embargo, esto no es suficiente para corroborar que $T(n,k) \in O(n^3 * k)$, ya que podría ser que $T(n,k) \in O(n^3 + k)$, mostrando el mismo comportamiento en la etapa de experimentación.