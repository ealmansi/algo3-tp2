Como dijimos anteriormente, hallar una solución al problema es equivalente a encontrar un camino de mínima cantidad de aristas entre el estado $(x_e,y_e,k_{max})$, representando al casillero de entrada con $k_{max}$ unidades de potencia extra, y cualquiera de los estados de salida. Es decir, alguno de los estados $(x_s,y_s,i)$ para cualquier $i \leq k_{max}$, dado que no importa la cantidad de unidades de potencia restantes al terminar.

Como sabemos, si realizamos BFS desde un nodo v en el grafo G, el mismo encuentra para cada nodo u $\in$ G un camino de mínimas aristas entre v y u. Por eso, al aplicar BFS en el grafo que modela la instancia del problema partiendo desde $(x_e,y_e,k_{max})$, encontramos el camino con mínima cantidad de aristas entre él y los estados de salida. Sin embargo, como se puede observar en la descripción del algoritmo, no es necesario computar explícitamente el camino hacia todos los estados de salida, sino que basta con tomar el camino hacia el primero de ellos que sea procesado en el ciclo del BFS. Dado que este algorítmo de búsqueda procesa todos los nodos por generaciones (primero el nodo inicial, luego sus descendientes, luego los descendientes de sus descendientes, etc.), cualquier estado de salida subsiguiente al primero en ser hallado tendrá un camino hacía el estado inicial con una cantidad de aristas mayor o igual a los encontrados previamente.