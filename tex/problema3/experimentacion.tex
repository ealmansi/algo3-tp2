\emph{Nota:} Como vamos a hablar mucho de las distintas podas vamos a denominarlas de la siguiente manera:
\begin{itemize}
\item Poda 1: Si aunque pongamos piezas en los casilleros que me quedan revisar, no llegamos a alcanzar la cantidad de piezas de la solución óptima encontrada a ese momento, corto.
\item Poda 2: Similar a la poda 1, pero además revisamos hasta $m$ casillas para adelante teniendo en cuenta las restricciones ya existentes en el tablero.
\end{itemize}

En este ejercicio, vamos a enfocar la experimentación orientándolo a la comparación entre las distintas podas, así como también comparar las mismas contra no usar ninguna poda. Cabe aclarar que no usar ninguna poda tiene un método de corte \emph{naïf}: cortar si logré llenar el tablero.

Primero probamos todos los casos posibles para n = 1 a n = 3 sin poda y para las dos podas que propusimos.

En los siguientes gráficos podemos observar que utilizar podas mejora drásticamente el tiempo de ejecución, respecto a no utilizar podas. No logramos realizar todos los casos de 3*3 sin podas ya que necesitaba un tiempo muy grande.

\begin{center}
  \begin{figure}[H]
    \includegraphics[width=0.5\linewidth]{problema3/graficos/comparacion_podas_todas_instancias_llamados.eps}
    \includegraphics[width=0.5\linewidth]{problema3/graficos/comparacion_podas_todas_instancias_tiempo.eps}
    \caption{Comparación con y sin podas. Izquierda: llamados recursivos, Derecha: tiempo}
    \label{fig:problema3-sin-y-con-podas}
  \end{figure}
\end{center}

En estos gráficos podemos observar que utilizar podas mejora tanto en la cantidad de llamados recursivos, así como también mejora en tiempo de ejecución. A su vez, si bien la cantidad de llamados recursivos que realiza el algoritmo al utilizar la poda 2 es menor a la cantidad de llamados recursivos que realiza al utilizar la poda 1, como el n es chico la cantidad de casos que poda no logra contrarestar el overhead que genera por lo que el tiempo de ejecución del algoritmo utilizando poda 2 es claramente mayor a utilizar la poda 1.

No usar podas es tan malo porque revisa todos los tableros posibles, y recién una vez revisados todos termina (salvando el caso en que justo encuentre el tablero completo). Esto implica que ya para tableros de 3*3 es un número grande de tableros posibles, y para tableros aún más grandes el algoritmo tarda una cantidad incomparable de tiempo con respecto a las podas. Por esta razón, sin podas no esta presente en el tercer gráfico de arriba, ni en los gráficos de abajo.

A continuacion experimentamos con 50 casos de 4*4 generados pseudo-aleatoriamente pero con una semilla constante. Cabe aclarar que se comparan los resultados de las dos podas para la misma entrada en cada caso.

\begin{center}
  \begin{figure}[H]
    \includegraphics[width=\linewidth]{problema3/graficos/comparacion_podas1y2_llamados_recursivos.eps}
    \caption{Comparación de podas, según llamados recursivos}
    \label{fig:problema3-podas-llamados}
  \end{figure}
\end{center}

\begin{center}
  \begin{figure}[H]
\includegraphics[width=\linewidth]{problema3/graficos/comparacion_podas1y2_tiempo.eps}
    \caption{Comparación de podas, según tiempo de ejecución}
    \label{fig:problema3-podas-tiempo}
  \end{figure}
\end{center}


En el gráfico \ref{fig:problema3-podas-llamados}, podemos observar claramente que la poda 2 logra podar mas casos y por consiguiente, realiza menos llamados recursivos. Vemos que al aumentar el n, la cantidad de casos que logra podar aumenta, logrando en la mayoria de los casos superar el overhead como se ve en la figura \ref{fig:problema3-podas-tiempo}. Sin embargo cabe aclarar que no siempre logra compensar lo suficiente el costo adicional de $O(m)$.

Como ejemplo, el caso 35 logra podar lo suficiente para lograr 4 veces menos llamados recursivos. Sin embargo, logra solamente ser un 25\% (aproximadamente) mejor en tiempo de ejecución.

Por otro lado, la poda 1 también tiene beneficios. Es considerablemente más fácil de implementar que la poda 2. A su vez, la poda 1 no depende de la cantidad de colores existentes, mientras que la poda 2 sí lo hace. La poda 2 crea un $vector<int>$ donde la cantidad de posiciones se corresponde con la cantidad de colores existentes. Si bien no es un problema muy grave, puede ocurrir que sí lo sea si tenemos una computadora con memoria acotada y tenemos un caso donde la cantidad de colores es muy alta, como por ejemplo un millón de colores.

Curiosamente, si bien utilizar la poda 1 incurre en una menor cantidad de tiempo de ejecución para casos mas chicos (como por ejemplo tableros de tamaño de 3*3), esto no se mantiene para casos más grandes (como tableros de tamaño 4*4). Es decir, la poda 2 comienza a tener más importancia cuando el tablero es más grande. Creemos que esto se debe a que las podas que realiza la poda 2 tienen más impacto con respecto al costo de realizar dicha poda en tableros de tamaño más grande.