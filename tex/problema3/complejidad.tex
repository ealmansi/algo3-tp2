Para calcular la complejidad vamos a tener en cuenta el peor caso, en el cual toda pieza disponible se puede agregar al tablero. Para facilitar el calculo llamaremos n a la cantidad de casillas, que en el problema esta representado por n*m.

%Facu: esto es lo que vi de como queda la sumatoria desarmando T(n,n) no se si estara bien igual
Si pensamos la recursión en una sola variable llegamos a una cota demasiado amplia:
$T(k)=n*T(k-1)$ donde n es la cantidad de fichas. Ya que no podemos asegurar que en cada paso tenemos disponible una pieza menos porque pudimos haber puesto una ficha blanca.

Por esto vamos a hacer una recursión en 2 variables, la cantidad de espacios por procesar y la cantidad de fichas disponibles. Siempre vamos a suponer que $n\leq k$, osea que siempre tengo una cantidad de fichas mayor o igual que de casilleros a rellenar.
Sabemos que: $T(n,k) = T(n-1,k) + k*T(n-1,k-1)$ ya que en cada paso ponemos una ficha blanca con lo cual se reduce el n pero no el k, y para las k fichas disponibles la agregamos al tablero y pasamos a un caso n-1,k-1.

Nuestro caso base es $(\forall k) T(0,k)= c$ siendo c una constante, ya que no tener casillas en el tablero por procesar significa que el problema terminó.

% Observando el arbol de recursion y pensando en el problema llegamos a la suposicion que:

Vamos a probar que:

$T(n,k) = c \sum_{i=0}^{n} \binom{n}{i} \binom{k}{i} i!$

Esto es lógico ya que $\binom{n}{i} \binom{k}{i} i!$ se puede pensar como la cantidad de formas de elegir i casillas entre n para poner fichas, i fichas entre k y ponerlas en todas sus permutaciones posibles. Si sumo para todo i entre 0 y la cantidad de casillas obtengo todas las combinaciones de fichas en el tablero posibles.

A continuación demostraremos haciendo inducción sobre n:

Caso base: $(\forall q)T(0,q)= c \sum_{i=0}^{0} \binom{0}{i} \binom{q}{i} i! = c \binom{0}{0} \binom{q}{0} 0! = c*1*1*1 = c$, siendo c constante. Es decir, $(\forall q)T(0,q)=O(1)$.

Paso Inductivo:

H.I.: $(\forall q)T(n-1,q) = c \sum_{i=0}^{n-1} \binom{n-1}{i} \binom{q}{i} i!$

q.v.q H.I. $\Rightarrow T(n,k) = c \sum_{i=0}^{n} \binom{n}{i} \binom{k}{i} i! = c \sum_{i=0}^{n} \frac{k!}{(k-i)!}\binom{n}{i}$ 

$(\forall k) T(n,k) = T(n-1,k) + k*T(n-1,k-1)$

$T(n,k) = c \sum_{i=0}^{n-1} \binom{n-1}{i} \binom{k}{i} i! + k * c \sum_{i=0}^{n-1} \binom{n-1}{i} \binom{k-1}{i} i!$ por H.I.

$ = c \sum_{i=0}^{n-1} \binom{n-1}{i} \frac{k!}{(k-i)!i!} i! + c \sum_{i=0}^{n-1} \binom{n-1}{i} \frac{k!}{(k-i-1)!i!}i! $ 

Ahora simplifico el i! en las dos sumatorias:

$ = c \sum_{i=0}^{n-1} \binom{n-1}{i} \frac{k!}{(k-i)!} + c \sum_{i=0}^{n-1} \binom{n-1}{i} \frac{k!}{(k-i-1)!} $

Luego saco el primer valor de la primera sumatoria y el último de la segunda:

$ = c (\binom{n-1}{0} \frac{k!}{(k-0)!} + \sum_{i=1}^{n-1} \binom{n-1}{i} \frac{k!}{(k-i)!} + \binom{n-1}{n-1} \frac{k!}{(k-(n-1)-1)!} + \sum_{i=0}^{n-2} \binom{n-1}{i} \frac{k!}{(k-i-1)!} )$

Como $\binom{n-1}{0} = \binom{n-1}{n-1} = 1$:

$ = c (\frac{k!}{(k-0)!} + \sum_{i=1}^{n-1} \binom{n-1}{i} \frac{k!}{(k-i)!} + \frac{k!}{(k-n)!} + \sum_{i=0}^{n-2} \binom{n-1}{i} \frac{k!}{(k-i-1)!} )$

Renombro i = j-1 en la segunda sumatoria:

$ = c (\frac{k!}{(k-0)!} + \sum_{i=1}^{n-1} \binom{n-1}{i} \frac{k!}{(k-i)!} + \frac{k!}{(k-n)!} + \sum_{j=1}^{n-1} \binom{n-1}{j-1} \frac{k!}{(k-j)!} )$ 

Como ambas sumatorias van de 1 a (n-1) puedo juntarlas:

$ = c (\frac{k!}{(k-0)!} + \frac{k!}{(k-n)!} + \sum_{i=1}^{n-1} ( \binom{n-1}{i} \frac{k!}{(k-i)!} + \binom{n-1}{i-1} \frac{k!}{(k-i)!} ) )$ 

Saco factor común $\frac{k!}{(k-i)!}$:

$ = c (\frac{k!}{(k-0)!} + \frac{k!}{(k-n)!} + \sum_{i=1}^{n-1} ( \frac{k!}{(k-i)!} (\binom{n-1}{i} + \binom{n-1}{i-1} )))$ 

Por propiedad de números combinatorios:

$ = c (\frac{k!}{(k-0)!} + \frac{k!}{(k-n)!} + \sum_{i=1}^{n-1} \frac{k!}{(k-i)!}\binom{n}{i})$ 

Multiplico por $\binom{n}{0} = \binom{n}{n}= 1$ a los dos primeros términos:

$ = c (\binom{n}{0}\frac{k!}{(k-0)!} + \binom{n}{n}\frac{k!}{(k-n)!} + \sum_{i=1}^{n-1} \frac{k!}{(k-i)!}\binom{n}{i})$

Para finalizar, agrego el elemento 0 y n de la sumatoria:

$ = c  \sum_{i=0}^{n} \frac{k!}{(k-i)!}\binom{n}{i}$ 

Entonces $T(n,k)  = c  \sum_{i=0}^{n} \frac{k!}{(k-i)!}\binom{n}{i}$ 

\begin{flushright}
    \hfill \ensuremath{\Box}
    \end{flushright}






% $T(n,n) = T(1, n) + n*n * T(1, n-1) + \binom{n}{2} * \binom{n}{2} * T(1, n-2) * 2 + \binom{n}{3}^2 *T(1,n-3) * 3! + \hdots + \binom{n}{i}^2 * T(1, n-i) * i! + \hdots + T(1,1) * n!$
% 
% Lo que hicimos en el paso anterior es contar todas las posibles soluciones, cada termino de la suma de arriba representa la cantiidad de formas de ponner $i$ fichas con $0 \leq i \leq n$. Para poner $i$ fichas en $i$ lugares del tablero, elegimos de $n$ posibles lugares, $i$ donde poner las fichas, y por cada elección, elegimos de $n$ posibles fichas, $i$ para distribuirlas, la cantidad de formas de hacer esas elecciones es $\binom{n}{i}^2$, y para cada combinación posible de $i$ fichas en $i$ espacios, tenemos que ver todas las permutaciones de las fichas en dichos espacios, esto es $i!$, entonces, la cantidad de formas posibles de ubicar $i$ fichas en $i$ espacios es de $i! * T(1,i) * \binom{n}{i}^2$, y por lo tanto:
% 
% $T(n,n) = \sum_{j=0}^{n}{j! * T(1,j) * \binom{n}{j}^2}$
% 
% Donde $T(1,j) \in O(j)$, entonces:
% 
% $T(n,n) = \sum_{j=0}^{n}{j! * j * \binom{n}{j}^2}$
% 
