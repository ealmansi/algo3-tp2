Nosotros afirmamos que nuestro algoritmo tiene una complejidad de $O(n^3 * k)$ operaciones. Vamos a demostrar esto de a partes.

Antes de empezar, vamos a hacer unas definiciones de tipo, válidas para los pseudocódigos de esta sección.
\begin{pseudo}
\State Tipo de dato Casillero es Tupla $\langle$ fila : entero, columna : entero, k : entero $\rangle$
\State Tipo de dato Entrada es Tupla $\langle$ n : entero, k\_max : entero, tablero : vector $\langle$ vector $\langle$ entero $\rangle$ $\rangle$, origen : Casillero, destino : Casillero $\rangle$
\State Tipo de dato Salida es Tupla $\langle$ saltos : entero, caminoMinimo : list $\langle$ Casillero$\rangle$ $\rangle$
\end{pseudo}

Desde un punto de vista mas ``macro'', nuestro algoritmo hace lo siguiente:

\begin{pseudo}
    \Procedure{Saltos en la \emph{Matrix}}{}
        \State Entrada E $\leftarrow$ leerEntrada() \Ode{n^2}
        \State Salida S(V) $\leftarrow$ resolver(E) \Ode{n^3*k}
        \State Imprimir(S) \Ode{n}
        \State \textbf{return} 0
    \EndProcedure
\end{pseudo}

\textbf{leerEntrada()} se encarga de cargar los datos vía \emph{stdin}. Por esto, realiza una cantidad constante de asignaciones todas las cuales son $O(1)$ ya que los tipos de datos son enteros o Casilleros (los cuales se componen de tres enteros). Una vez cargados los datos, creamos el tablero. El tablero es vector $\langle$ vector $\langle$ entero $\rangle$ $\rangle$ que representa el campo de juego. Por esto, tiene tamaño $n*n$ y realiza justamente $n^2$ asignaciones de enteros. Sumando todo esto, \textbf{leerEntrada()} tiene un costo de $O(n^2)$.

Sin embargo, cabe aclarar que después realizamos una asignación (ya que la Entrada E no es pasada por referencia). Esta asignación de Entrada se compone de tres asignaciones de enteros ($O(1)$ cada una), un asignación de tablero ($O(n^2)$) y dos asignaciones de Casillero ($O(1)$ cada uno). Pero esto no cambia el costo total de $O(n^2)$.

\textbf{resolver(E)} es el núcleo de nuestro algoritmo y es lo que más cuesta, y por esta razón lo vamos analizar con más detalle.

\begin{pseudo}
\State Tipo de dato Matriz3D es vector $\langle$ vector $\langle$ vector $\langle$ Tupla $\langle$ Casillero, k $\rangle$ $\rangle$ $\rangle$ $\rangle$
    \Procedure{resolver}{$Entrada E$}
        \State Matriz3D yaPase $\leftarrow$ crearMatriz() \Ode{1}
		\Comment{Empieza BFS, pero modificado para nuestra conveniencia}
		\State q $\leftarrow$ crearCola() \Ode{1}
		\Comment{La funcion que sigue marca al casillero e.origen en yaPase con cantidad de saltos 0}
		\State marcoComo(e.origen, yaPase, 0) \Ode{1}
		
		\State encolar(e.origen, q) \Ode{1}
		
		\While{!q.empty()} \Ode{1}
			\State c $\leftarrow$ q.desencolar() \Ode{1}
			\If{c == destino}
				\State break
			\EndIf
			
			\For{cada Casillero u en losAdyacentes(e,c)} \Ode{n}
			    \If{!estaEn(u,yaPase)} \Ode{1}
					\State marcoComo(u, yaPase, cantSaltos(c) + 1) \Ode{1}
					\State encolar(u, q) \Ode{1}
				\EndIf
      		\EndFor
		\EndWhile

      	\Comment{Terminamos BFS y pasamos a escribir la salida. escriboSalida en nuestro codigo no es una funcion aparte, pero para este pseudocodigo lo decidimos dejar así.}
      	\State escriboSalida(S, yaPase) \Ode{n}
      	\State \textbf{return} S
        
    \EndProcedure
\end{pseudo}

La parte importante de resolver, es el BFS que realizamos, ya que las otras partes del algoritmo tienen complejidad menor.

Vamos a empezar diciendo que un casillero cualquiera puede tener a lo sumo $2*(n-1)$ casilleros adyacentes (toda su fila y toda su columna, menos él mismo). Por esto, cuando realizamos losAdyacentes, buscamos a lo sumo $O(n)$ adyacentes. Si ya sabemos que estamos fuera del rango del tablero, no vale la pena seguir buscando. Por esta misma razón, el ciclo for que utiliza los adyacentes corre a lo sumo $O(n)$ veces por cada casillero.

Teniendo en cuenta esto, vamos a correr el ciclo while grande a lo sumo una vez por casillero. Cabe recordar que tenemos $n*n*k$ casilleros en total: un casillero por cada celda del tablero, y ese tablero lo tenemos k veces. Lo peor que puede suceder es tener que correr el while para cada casillero, por lo que tendríamos un costo total de $O(n*n*n*k)$, que es igual a $O(n^3*k)$.

Al escribir la salida, escribimos la cantidad de saltos, y cuáles fueron dichos saltos, osea cuál fue el camino que se tomó desde el inicio hasta el dstino. Podemos observar que la cantidad de saltos es $O(n)$. En el peor caso, todos los casilleros tienen potencia 1 (lo menor que pueden valer) y el jugador no tiene potencias extras para gastar ($k=0$). En este caso, estamos en el primer ejemplo de la sección  \ref{problema3-descripcion}. En este caso, una solución óptima es recorrer desde la fila origen hasta la fila destino, y luego de la columna origen hasta la columna destino. Tanto las filas como las columnas pueden estar separadas a lo sumo por $n-1$ casilleros. Por esto, el camino más corto esta claramente acotado por $2*n$, por lo que tenemos a lo sumo $O(2*n)$ iteraciones de ciclo. Por todo esto, podemos afirmar que \textbf{Imprimir(S)} tiene una complejidad de $O(n)$

Finalmente, \textbf{Imprimir(S)} es la función encargada de imprimir el resultado por pantalla. Vamos a utilizar un pseudocodigo para probar su complejidad.

\begin{pseudo}
    \Procedure{Imprimir}{$Salida s$}
        \State imprimir(s.saltos) \Ode{1}

		\For{cada Casillero c en s.caminoMinimo} \Ode{1}
        \State imprimir(c.fila) \Ode{1}
        \State imprimir(c.columna) \Ode{1}
        \State imprimir(kUsado(c, padre(c))) \Ode{1}
        \Comment{Este kUsado en nuestro algoritmo en C++ lo obtenemos sabiendo cual era nuestro padre con un iterador auxiliar. Como esto es pseudocodigo decidimos dejarlo de esta manera}
      	\EndFor
        
    \EndProcedure
\end{pseudo}

Como se puede observar, todas las impresiones por pantalla son $O(1)$ y la guarda del for también cuesta $O(1)$. Sin embargo, falta justificar cuantas veces corre el ciclo for. Como probamos anteriormente, el camino está acotado por $2*n$. Por esto, realizamos $O(n)$ cantidad de ciclos for.

Sumando todas las complejidades, llegamos a obtener una complejidad temporal total de $O(n^3 * k)$, con lo que cumplimos con la complejidad temporal pedida.