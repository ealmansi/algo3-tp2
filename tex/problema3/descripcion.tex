Este problema: plantea la siguiente situación, se tiene un tablero de dimensiones $n*m$ y la misma cantidad de piezas y el objetivo es insertar la mayor cantidad posible de piezas en dicho tablero. Las piezas son cuadradas y tienen un color en el lado superior, uno en el lado inferior, uno en el lado derecho y uno en el lado izquierdo. Una pieza se puede colocar adyacente a otra sólo si sus lados coinciden (por ejemplo si una pieza tiene el lado izquierdo de color rojo, a su izquierda sólo se puede colocar una pieza cuyo lado derecho también sea rojo). La otra variable que tiene este problema es la cantidad de colores posibles.

Por ejemplo supongamos que tenemos un tablero de $2*2$ y que hay 4 colores disponibles: Amarillo, Azul, Rojo y Verde. Supongamos, además, que contamos con las siguientes piezas:
\begin{center}
\begin{tabular}{c|c|c|c|c}
 Pieza & Izq & Der & Sup & Inf \\
 \hline
 1 & Amarillo & Azul & Rojo & Verde \\
 2 & Rojo & Azul & Verde & Amarillo \\
 3 & Azul & Amarillo & Azul & Verde \\
 4 & Amarillo & Rojo & Verde & Azul \\
\end{tabular}
\end{center}

Una posible solución (que pondría todas las fichas en el tablero) de esta instancia del problema sería:
\begin{center}
\begin{tabular}{|c|c|}
  \hline
  1 & 3 \\
  \hline
  4 & 2 \\
  \hline  
\end{tabular}
\end{center}
%Despuès vemos si es mejor centrar las tablas o no