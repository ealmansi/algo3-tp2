Este problema trata de resolver para un participante un camino óptimo en un gran campo de juego cuadrado. Este campo está dividido en celdas que forman una matriz de $n * n$. En cada celda hay un resorte propulsor que cada participante puede usar para saltar hacia otras celdas del campo. Cada resorte tiene un cierto número de unidades de potencia máximas $p$ que indica la cantidad de celdas que se pueden atravesar saltando desde este resorte. Sólo se puede saltar hacia adelante, atrás, izquierda o derecha, no se puede saltar en diagonal.

Además, cada participante cuenta desde el inicio con $k$ unidades extra de potencia para los saltos. Antes de cada salto, cada participante puede elegir usar algunas de esas unidades para potenciar más aun el salto que va a efectuar. Las unidades extra utilizadas se irán descontando de las $k$ con las que empieza cada participante. No hace falta que el participante gaste todas sus unidades de potencia extra al terminar el juego pero tampoco se beneficia si termina con unidades de sobra.

Por ejemplo supongamos que tenemos un tablero de $5*5$, $k$ = 0, origen en (1,1) y destino en (5,5). Asumimos que la celda ($x$, $y$) se encuentra en la fila $x$ y la columna $y$. También asumimos que (1,1) se encuentra en la esquina superior izquierda, mientras que la celda (5,5) se encuentra en la celda inferior derecha. Cada celda está representada por su potencia.
\begin{center}
\begin{tabular}{|c|c|c|c|c|}
\hline
 1 & 1 & 1 & 1 & 1 \\
 \hline
 1 & 1 & 1 & 1 & 1 \\
 \hline
 1 & 1 & 1 & 1 & 1 \\
 \hline
 1 & 1 & 1 & 1 & 1 \\
 \hline
\end{tabular}
\end{center}

Una posible solución de esta instancia del problema sería: (1,1), (1,2), (1,3), (1,4), (1,5), (2,5), (3,5), (4,5), (5,5).

Utilizando el mismo caso, pero con $k$ = 6, una posible solución sería: (1,1), (1,5), (5,5). Cabe notar que utilizar un $k >$ 6 no afecta la solución ya que esas unidades extra no las podemos aprovechar.

Sin embargo, no hace falta que todas las celdas tengan la misma potencia. Otro tablero, que también use origen en (1,1) y destino en (5,5) podría ser el siguiente:
\begin{center}
\begin{tabular}{|c|c|c|c|c|}
\hline
 4 & 3 & 2 & 1 & 4 \\
 \hline
 3 & 3 & 2 & 1 & 1 \\
 \hline
 2 & 2 & 2 & 1 & 1 \\
 \hline
 1 & 1 & 1 & 1 & 1 \\
 \hline
\end{tabular}
\end{center}

Una posible solución sería: (1,1), (1,5), (5,5), sin importar el $k$ con el que se empieza ya que no nos hace falta.

Cabe notar que a veces conviene alejarse del destino, si esto resulta en menos saltos totales. Por ejemplo, usando $k$ = 0, origen en (1,2) y destino en (5,5):
\begin{center}
\begin{tabular}{|c|c|c|c|c|}
\hline
 4 & 1 & 1 & 1 & 1 \\
 \hline
\end{tabular}
\end{center}

Una posible solución de esta instancia del problema sería: (1,2), (1,1), (1,5).