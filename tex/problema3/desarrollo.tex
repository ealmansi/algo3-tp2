Para resolver este problema, lo modelamos de la siguiente manera:

$\bullet$ Cada estado posible en el juego es estar en una casilla dada con cierta cantidad de unidades de potencia, esto lo representamos con (int x,int y, int k) que representa estar posicionado en la casilla de la fila x y la columna y, con k unidades de potencia para usar.

$\bullet$ Dada la cantidad de filas y columnas n y la cantidad de unidades de potencia máxima $k\_max$, a priori los estados posibles son (x,y,k) / $0 < x,y \leq n \wedge 0 \leq k \leq k\_max$. 

$\bullet$ Moverme al este representa incrementar la columna, al oeste decrementarla, al norte incrementar la fila y al sur decrementarla.

$\bullet$ Dado un estado $e_i = (x_0,y_0,k_0)$ se puede saltar sin consumir unidades de potencia, a un estado con el mismo k, pero con alguna de mis coordenadas x o y cambiadas entre 1 y la potencia de la casilla en la que me encuentro. Osea  $(\forall 0 < j \leq p ) e_i$ puede saltar a $(x_0,y_0+j,k_0), (x_0,y_0-j,k_0), (x_0+j,y_0,k_0), (x_0-j,y_0,k_0)$.

$\bullet$ Dado un estado $e_i = (x_0,y_0,k_0)$ también se puede saltar consumiendo alguna cantidad de potencia entre 1 y k, de esta manera se saltaría a un estado con k igual a $k_0$ menos la cantidad de potencia gastada y cambiando el x e y en alguna dirección una cantidad de unidades igual a la potencia máxima de la casilla en la que me encuentro más la cantidad de unidades de potencia gastadas. Osea $(\forall 0 < j \leq k_0 ) e_i$ puede saltar a $(x_0,y_0+p+j,k_0-j), (x_0,y_0-p-j,k_0-j), (x_0+p+j,y_0,k_0-j), (x_0-p-j,y_0,k_0-j)$. 

En ambos casos solo tendremos en cuenta los estados posibles vistos en la condición anterior (que su x e y no salgan del rango [1,n]).

Queremos encontrar la forma de llegar desde la casilla de entrada $x_e,y_e$ hasta la casilla de salida $x_s,y_s$ con una cantidad de unidades de potencia igual a k\_max en la mínima cantidad de saltos posibles. En nuestro modelo seria partiendo del estado $(x_e,y_e,k\_max)$ queremos llegar al estado $(x_s,y_s,i)$ para algún i entre 0 y k\_max en la mínima cantidad de saltos. Como en nuestro modelo cada cambio de estado equivale a un salto queremos hallar un camino mínimo entre $(x_e,y_e,k\_max)$ y $(x_s,y_s,i)$ para algún i entre 0 y k\_max en nuestro dígrafo donde los vértices son los posibles estados y las incidencias están dadas si se puede saltar de un estado $e_i$ a uno $e_j$ .

Para encontrar esto usamos BFS terminando la primera vez que encontramos un vértice $(x_s,y_s,i)$.



